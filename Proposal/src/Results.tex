%The funded project is expected to . 











This section talks about the innovation of the proposal and impact in terms of relevance of the progress in basic research for the relevant scientific community. 
The proposed research on fairness %and addressing adversarial attacks 
in recommender systems for software engineering has the potential to yield significant outcomes, %. %, which can have far-reaching effects on both academia and industry. 
%Our proposed approach %extend beyond mere security enhancements, and 
%has the potential to 
positively impacting the software engineering field, and artificial intelligence.%, user privacy, and the broader software industry. 
The findings and defense mechanisms developed in this research will be disseminated through academic publications and presentations, contributing to the body of knowledge in software engineering. % security and adversarial machine learning. %This section elaborates on the implications of our work, as well as the prospective publications. % as follows.
%\vspace{-.4cm}




\subsection{Novelty}



%\textbf{Expected contributions.} %If the proposal is funded, % by %securing funding from 
%VIASM, %my envisioned contributions entail the establishment of a %dedicated 
%I will %collaborate with my colleagues at the School of Information and Communications Technology (SOICT), Hanoi University of Science and Technology (HUST),
%research team at the University of L'Aquila, 
%specifically focused on addressing crucial challenges within the domain. We 
%invest efforts to conduct %comprehensive 
%research to identify potential threats and devise effective countermeasures~\cite{10.1007/978-3-030-49461-2_18}. 
We aim to empower RSSEs %Recommender Systems for Software Engineering (RSSEs) 
with the capability to effectively deal with popularity bias, while maintaining or even enhancing their accuracy. Our work distinguishes itself from state-of-the-art research in the following aspects.
%The main contributions are as follows: %stands out from existing research in the following key aspects: % key ways:

\begin{itemize}
	
	\item \textbf{Pioneering exploration.} We take the initiative to bring attention to the previously overlooked issues of popularity bias affecting RSSEs. This represents a pioneering effort in the field.
	
	\item \textbf{Holistic approach.} We propose a holistic and integrated approach to fortify RSSEs, leveraging tailored reinforcement learning. This comprehensive strategy is designed to enhance their fairness. %and resilience.
	
	\item \textbf{Applicability.} The techniques we develop are not limited to RSSEs only, they can be readily applied to fine-tune pre-trained deep learning models in the broader context of software engineering, contributing to the advancement of AI-driven solutions in the field. The potential of Large Language Models (LLMs) in software engineering has been recently acknowledged~\cite{10109345}. However, these models are trained on vast and diverse datasets, which can introduce biases. Additionally, LLMs often require prompt tuning to tailor them to specific purposes, necessitating fine-tuning with relevant data. 
	In this context, the techniques developed in this research project offer practical solutions by mitigating biases within training data for LLMs. %For the software engineering industry, our approach and research outcomes hold substantial promise, \ie they can be readily adopted by software companies, seamlessly integrated into their recommender systems. This adoption promotes a culture of fairness and inclusivity within the industry. %Furthermore, it underscores a commitment to safeguarding users from potential adversarial threats. 
	%By addressing these critical issues, our research significantly enhances the applicability of Large Language Models in software engineering, making them more robust, ethical, and aligned with industry standards.
	
\end{itemize}



%With our contributions, 
%We aim to advance the state of knowledge in software engineering and AI ethics while offering practical solutions that can have a significant impact on the industry, research community, and society at large.

%\vspace{-.4cm}



\subsection{Implications}

%These implications are outlined below:


%The proposed research on fairness in recommender systems for software engineering has the potential to yield several significant implications and outcomes, which can benefit both academia and industry. These implications are outlined below:

%The implications of 


%=========================

\begin{itemize}
	\item \textbf{Enhancing user trust and satisfaction.} By mitigating bias and promoting fairness in recommender systems, our approach aims to enhance user trust and satisfaction with software engineering platforms. %This will help to improve user engagement and loyalty. %Enhanced System Security: 
	%With the conceived methods to defend RSSEs against adversarial attacks, our research can enhance the security and robustness of recommender systems used in software engineering platforms. This is crucial for safeguarding sensitive data and ensuring system integrity. %Preservation of User Trust: 
	As users rely on recommendations to make critical decisions in software engineering, our research helps to preserve user trust by ensuring that recommendations remain trustworthy.

	\item \textbf{Improving system performance.} Fairness-aware recommender systems are likely to generate recommendations that are not only unbiased but also more accurate and relevant to users. This improvement in recommendation quality can positively impact the overall performance of software engineering platforms. %The ability to thwart adversarial attacks can help maintain the reliability of recommender systems, ensuring that users receive dependable recommendations even in the presence of malicious actors.

	\item \textbf{Ethical and responsible AI development.} Addressing fairness issues in recommender systems aligns with the principles of ethical and responsible AI development. Our research contributes to the responsible use of AI in software engineering by reducing potential biases and discrimination. %Protection of User Privacy: 
%	Addressing adversarial attacks can lead to improved protection of user privacy within recommender systems. This is particularly important in software engineering, where proprietary and confidential information is frequently handled.
 
%%	\item \textbf{Industry Adoption.} 
%	\item \textbf{Applicability.} 
%%	The recent years witness a proliferation of Large Language Models (LLMs). 
%	The potential of Large Language Models (LLMs) in Software Engineering has been considered to be enoumous~\cite{10109345}. %Among others, ... still containing bias. 
%	Nevertheless, such models are trained with a lot of data coming from different sources, and among other issues, bias and adversarial attacks. LLMs still rely on prompt tuning to function, \ie to be tailored for a specific purpose, they need to be fine tuned with proper data. In this respect, the techniques conceived from this research project can come in handy, defusing bias, and detecting adversarial intents in the training data. % can be used.
%	For industry, our approach and research outcomes can be adopted by software engineering companies, and integrated into their recommender systems. This will foster a culture of fairness and inclusivity in the industry, as well as the attitude towards the protection of users from adversarial threats. %Our defense mechanisms and research outcomes can be adopted by software engineering companies, enhancing the security of their recommender systems and protecting their users from adversarial threats.


	

	
	\item \textbf{Educational impact.} The ethical implications of AI, including bias mitigation and adversarial counteraction, have gained significant attention in recent years. Our research offers an opportunity to educate future AI professionals about the importance of ethical considerations when developing AI-based systems. It can be integrated into AI ethics courses to promote responsible AI development. %Essentially, the developed fairness-aware attack-proof recommender systems can serve as learning materials, % for software engineering and AI courses, %promoting responsible AI development among students, %future professionals, %. The conceived defense strategies against adversarial attacks can be used as educational material for software engineering and cybersecurity courses, 
	%providing students with the knowledge and skills to protect recommender systems.
	
%	\item \textbf{Social impact.} After the devastating earthquake in L'Aquila in 2009,\footnote{L'Aquila Earthquake 2009 \url{https://www.internetgeography.net/topics/laquila-earthquake-2009/}} several efforts have been made to revitalize the city. Among others, investment in education and research is the key contributing factor, making the city become accessible to more people.
%	%The foundation of the Gran Sasso Science Institute. 
%	At the University of L'Aquila, we are committed to do cutting-edge research in various disciplines, including software engineering. 
%%	We will contribute to. 
%	Having a project funded by Sony will allow us %the University of L'Aquila 
%	to attract more external/foreign researchers to come to, and work in L'Aquila. More importantly, doing pioneering research %in the fairness 
%	in software engineering will help the university and thus, the city, increase their ranking and visibility. % of the University, as well as the city.
	
	
%	This will help  

%	\item \textbf{Interdisciplinary Collaboration.} This research may also encourage interdisciplinary collaboration between software engineers, data scientists, ethicists, and legal experts, fostering a holistic approach to addressing fairness issues, and adversarial challenges.
\end{itemize}
%=========================
\vspace{-.4cm}



%Maintenance of System Reliability: The ability to thwart adversarial attacks can help maintain the reliability of recommender systems, ensuring that users receive dependable recommendations even in the presence of malicious actors.

%Reduced Economic Losses: Adversarial attacks can have significant economic consequences, including financial losses due to system breaches and fraudulent activities. Our research can contribute to reducing these economic losses.

%Legal and Regulatory Compliance: In many jurisdictions, data protection regulations require organizations to implement security measures to protect user data. Our research can assist software engineering platforms in complying with these regulations.

%Knowledge Dissemination: The findings and defense mechanisms developed in this research will be disseminated through academic publications and presentations, contributing to the body of knowledge in software engineering security and adversarial machine learning.

%Practical Guidelines: We anticipate that our research will lead to the formulation of practical guidelines and best practices for defending recommender systems against adversarial attacks, benefiting software engineering practitioners and organizations.

%Interdisciplinary Collaboration: This research may foster interdisciplinary collaboration between software engineers, cybersecurity experts, and machine learning researchers, facilitating a comprehensive approach to addressing adversarial challenges.

%In conclusion, the implications of our proposed approach extend beyond mere security enhancements and have the potential to positively impact the software engineering field, user privacy, and the broader software industry. %The research aims to contribute to the resilience and trustworthiness of recommender systems used in software engineering contexts, ultimately safeguarding users and valuable software assets.


%=========================





%Recommender systems, as one of the most widespread applications of machine learning, play a significant role in aiding human decision-making processes. The quality of the recommendations they generate is closely linked to user satisfaction and the interests of the platforms. However, because recommender systems heavily rely on data and algorithms, they can be susceptible to biases that may lead to unfair outcomes, potentially eroding trust in the systems. Consequently, it is imperative to address fairness issues in recommendation scenarios. %Recent attention has been increasingly focused on considerations of fairness within recommender systems, resulting in a growing body of literature dedicated to methods aimed at promoting fairness in recommendations.



%Fairness is an ongoing topic in the domain of Recommender Systems.
%
%Our aim is to develop suitable techniques to tackle popularity bias, and adversarial threats to RSSEs. The implication is ... Systems usually provide very popular libraries to developers. 
%
%We are the first to raise the issue of popularity bias, and adversarial attacks to RSSEs. We will propose a holistic approach to make RSSEs fair and attack-proof, employing and tailoring reinforcement learning, and adversarial training for this purpose. The conceived techniques can be applied to fine tune pre-trained deep learning models in Software Engineering.%applied in real-world recomm
%







%Through RQ$_2$, we noticed that the characteristics of datasets are a contributing factor in the ability of \LS to mitigate popularity bias. In fact, DS$_3$ contains a considerably large number of projects (56,091), but only a small number of libraries (762). In contrast, compared to DS$_3$, DS$_2$ has a lower number of projects (5,200), but its number of libraries is substantially higher (31,817). In this respect, the distribution of the libraries across the projects in DS$_3$ is much denser than that of DS$_2$, as the former features projects from the same ecosystem (Android). It is then necessary to perform an in-depth analysis of how the characteristics of a dataset impact on the ability of TPL recommender systems to deal with popularity bias. % On the other hand, \LS attempts to mitigate bias by increasing the relevance of unpopular libraries with an adaptive weighting scheme.

% researchers
%This has impact on both 
%=====================
%\textbf{Popularity bias.} 
%Chakraborty \etal \cite{ChakrabortyM0M20} introduced an approach that incorporates both pre-processing (\ie before training) and in-processing (\ie during training).
%=====================


\subsection{Prospective publications}


The findings and methodologies conceived in this funded project will contribute to state-of-the-art research in software engineering and AI ethics. %be disseminated through academic publications and presentations. %, contributing to the body of knowledge in software engineering and AI ethics. %The results of the funded project would be 
We will target both the software engineering, and the machine learning communities, aiming to have \emph{at least two articles} submitted and accepted for publication to a Rank A or A* conference,\footnote{CORE Rankings Portal~\url{http://portal.core.edu.au/conf-ranks/}} or %and \emph{two articles} submitted and accepted for publication to 
Scimago Q1 journals.\footnote{Scimago Journal \& Country Rank \url{https://www.scimagojr.com/}} %We target %plan to submit our work to 
The following venues are considered: Int. Conf. on Automated Software Engineering (ASE, Rank A*); Int. Conf. on Software Engineering (ICSE, Rank A*); Int. Conf. on Evaluation and Assessment in Software Engineering (EASE, Rank A); Int. Conf. on Mining Software Repositories (MSR, Rank A); The ACM Conf. on Recommender Systems (RecSys, Rank A); Int. Conf. on Software Analysis, Evolution, and Reengineering (SANER, Rank A); The ACM Int. Conf. on Information and Knowledge Management (CIKM, Rank A); Elsevier Information and Software Technology Journal (IST, Q1); Elsevier Journal of Systems and Software (JSS, Q1); Elsevier Expert Systems with Applications (ESWA, Q1); IEEE Transactions on Software Engineering (TSE, Q1).

\vspace{-.4cm}


%one of the following journals:

%\begin{itemize}
%	\item[--] International Conference on Automated Software Engineering (ASE), Rank A*.
%	\item[--] International Conference on Software Engineering (ICSE), Rank A*.
%	\item[--] International Conference on Evaluation and Assessment in Software Engineering (EASE), Rank A.
%	\item[--] International Conference on Mining Software Repositories (MSR), Rank A.
%	\item[--] The ACM Conference on Recommender Systems (RecSys), Rank A.
%	\item[--] International Conference on Software Analysis, Evolution, and Reengineering (SANER), Rank A. 
%	\item[--] Elsevier Information and Software Technology Journal (IST), Q1.
%	\item[--] Elsevier Journal of Systems and Software (JSS), Q1.
%	\item[--] IEEE Transactions on Software Engineering (TSE), Q1.
%\end{itemize}




%\vspace{-.2cm}

%\textbf{TODO: Add some milestones here ...}



%\subsection{Budget plan} \label{sec:BudgetPlan}
%
%A tentative budget plan is shown in Table~\ref{tab:Budget}. The award will be completely used for the research activities pertinent to the project. The largest part of the money will be paid two postdoctoral researchers for the duration of one year. We also plan to spend money for the purchase of equipment, including servers and laptops for running the experiments. Moreover, a certain amount of the budget is reserved for registration fees and travel expense for conferences and meetings. %
%
%%\vspace{-.4cm}
%\begin{table}[h!]
%	\centering
%	\vspace{-.2cm}
%	%	\scriptsize	
%	\footnotesize
%	\scriptsize
%	\caption{Budget distribution.}
%	\begin{tabular}{| p{0.4cm}|p{6.6cm} | l | p{1.0cm} | l  | p{3.4cm} |}
	%		\hline
	%		\textbf{No.}  & \textbf{Item} &  \textbf{Price (USD)} &  \textbf{Quantity} & \textbf{Amount (USD)} & \textbf{Description}  \\ \hline
	%		1  & Paying salary for two full-time postdoctoral researchers to work on the project & 35,000 & 2 & 70,000 & Both researchers are recruited for one year \\ \hline
	%		2 & Acquisition of devices, including GPU computers, laptops, monitors, keyboards, and mouses & 20,000 & 1 & 20,000 & The computers, and devices will be used for other research activities once the project is finished \\ \hline
	%		3 & Registration fee and travel expenses for conferences/meetings & 5,000 & 2 & 10,000 & Fees for at least two co-authors to attend conferences \\ \hline
	%%		3  & A computer with GPU to run Deep Learning algorithms & 15,000 & 1 & 15,000 & \multirow{4}{12.5em}{The computers, and devices will be used for other research activities once the project is finished}  \\ \cline{1-5}
	%%		4  & Dell XPS 13 laptop & 2,000 & 2 & 4,000 &  \\ \cline{1-5} %\hline
	%%		5  & Dell Curved USB-C Hub Monitor U3821DW 37.52"  & 1,200  &  2 & 2,400  &   \\ \cline{1-5} 
	%%		6  & Dell Multi-Device Wireless Keyboard and Mouse Combo KM7120W  & 103  & 2  & 206  &   \\ \hline 
	%%		7  &   &   &   &   &   \\ \hline
	%%		8  &   &   &   &   &   \\ \hline  
	%%		9  &   &   &   &   &   \\ \hline 
	%%		10  &   &   &   &   &   \\ \hline 
	%		  \multicolumn{2}{l|}{}   & \multicolumn{2}{c|}{\textbf{Total (USD)}} & \textbf{100,000} &  \\ \cline{3-6}		
	%	\end{tabular}
%	\vspace{-.4cm}
%	\label{tab:Budget}
%\end{table}
%
%
%\subsection{Force majeure clause} \label{sec:ForceMajeureClause}
%
%We anticipate that there exist unforeseeable events that might affect the project's timeline or completion. In particular, we may %it may be the case that we 
%encounter difficulties in recruiting the %necessary 
%personnel required for the successful execution of this research project. To mitigate the risk, the Principal Investigator (PI) shall make diligent and continuous efforts to recruit the required personnel as per the project plan and budget. These efforts shall include but not be limited to advertising positions, conducting interviews, and actively seeking suitable candidates. Our priority is to look for external applicants, \ie those that come from different countries/institutions, or the neighbour Gran Sasso Science Institute (GSSI),\footnote{GSSI is also located in the City of L'Aquila, thus easing the recruitment \url{http://gssi.it/}} so as to avoid the so-called \emph{intellectual inbreeding}. However, in case this is not possible, %to secure the recruitment, 
%we will resort to considering our graduates, \ie those that recently got a Ph.D. in Computer Science from the University of L'Aquila for the open positions. %,  institution.
%
%If the force majeure event significantly affects the project's timeline, we will discuss with Sony %both parties agree 
%to extend project milestones and deadlines by a duration commensurate with the duration of the force majeure event. The new timeline shall be agreed upon in writing between both parties.%, \ie Sony and our group. Both parties shall make reasonable efforts to mitigate the impact of the force majeure event and to resume the project activities as soon as possible. %practicable.

%In the event that the PI anticipates or encounters difficulties in recruiting the required personnel, the PI shall promptly notify the Funding Agency in writing, providing details regarding the challenges faced, potential delays, and proposed solutions. Upon notification of recruitment challenges, the PI and the Funding Agency shall engage in a collaborative discussion to explore possible modifications to the project plan, including adjustments to project scope, objectives, timelines, and budget, to mitigate the impact of the personnel shortage.
%\vspace{-.4cm}