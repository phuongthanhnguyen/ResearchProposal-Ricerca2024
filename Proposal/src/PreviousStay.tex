
From 21st July 2022 to 31st August 2022, I had a research stay funded by VIASM. During that time, I worked on a project about adversarial attacks to API recommender systems. %using deep learning. 
The results of this work have been published in the following paper (with acknowledgment to VIASM).

\begin{itemize}
	\item %\bibitem{NguyenESWALUPE}  %\textbf{Journal manuscript}:\\  %\textbf{Journal paper}:\\	
	\underline{Phuong T. Nguyen}, Claudio Di Sipio, Juri Di Rocco, Riccardo Rubei, Davide Di Ruscio$^{*}$, Massimiliano Di Penta ``\emph{Fitting Missing API Puzzles with Machine Translation Techniques},'' Elsevier Expert Systems with Applications (ESWA), 2023, ISSN: 0957-4174, DOI: \href{https://doi.org/10.1016/j.eswa.2022.119477}{https://doi.org/10.1016/j.eswa.2022.119477}. 
\end{itemize}

During the research stay, I established collaborations with two research groups in Hanoi. In particular, I have been working with Dr. Mai Anh Bui\footnote{\url{https://soict.hust.edu.vn/ts-bui-thi-mai-anh.html}} (HUST) and her students to develop recommender systems for software engineering. So far, our results have been reported in different papers, and some of them are now under review by various conferences and journals, including the 28th International Conference on Evaluation and Assessment in Software Engineering (EASE 2024), and the Journal of Systems and Software. Among them, the following paper has been accepted for publication.

\begin{itemize}
	\item Anh Ho, Anh M. T. Bui, \underline{Phuong T. Nguyen}, Amleto Di Salle. ``\emph{Fusion of deep convolutional and LSTM recurrent neural networks for automated detection of code smells},'' in Proceedings of the International Conference on Evaluation and Assessment in Software Engineering 2023 (EASE 2023), DOI: \href{https://doi.org/10.1145/3593434.359347}{https://doi.org/10\-.1145/3593434.3593478}.	
\end{itemize}

I have had collaboration with a research group at VNU University of Engineering and Technology (UET), and two papers have been published as follows:



\begin{itemize}
	\item Thu T. H. Doan, \underline{Phuong T. Nguyen}, Juri Di Rocco, Davide Di Ruscio. ``\emph{Too long; didn’t read: Automatic summarization of GitHub
		README.MD with Transformers},'' in Proceedings of the International Conference on Evaluation and Assessment in Software Engineering 2023 (EASE 2023), DOI: \href{https://doi.org/10.1145/3593434.3593448}{https://doi.org/10.11\-45/3593434.3593448}.
	\item 	Linh T. Duong, Thu T. H. Doan, Cong Q. Chu, \underline{Phuong T. Nguyen}$^{*}$, ``\emph{Fusion of edge detection and graph neural networks to classifying electrocardiogram signals},'' Elsevier Expert Systems with Applications (ESWA), 2023, ISSN: 0957-4174, DOI: \href{https://doi.org/10.1016/j.eswa.2023.120107}{https://doi.org/10.1016/j.eswa.2023.120107}.
\end{itemize}


Under my supervision, two students, one from HUST, and the other one from UET, already won a competition for full scholarships with the Grans Sasso Science Institute (GSSI), Italy. Both of them are now enrolled as first year students at GSSI.\footnote{\url{https://gssi.it/people/students/students-computer-science}} Moreover, another student from HUST has just been offered a full PhD scholarship at the University of Melbourne, Australia, and he is going to be enrolled soon in the program. 


If my research stay is funded by VIASM, then I will be able to further strengthen the collaborations, as well as to bring more talented students to Italy to do research with me. 

%This will allow me to.