%%%%%%%%%%%%%%%%%%%%%%%%%%%%%%%%%%%%%%%%%
% CV template for ProDigital proposal
%
% Author: Martin Hoelzer
%
% Important note:
% This template needs to be compiled with XeLaTeX.
% The main document font is called Fontin and can be downloaded for free
% from here: http://www.exljbris.com/fontin.html
%
%%%%%%%%%%%%%%%%%%%%%%%%%%%%%%%%%%%%%%%%%

%----------------------------------------------------------------------------------------
%	PACKAGES AND OTHER DOCUMENT CONFIGURATIONS - JUST LEAVE THIS AT IS IS
%----------------------------------------------------------------------------------------
\documentclass[a4paper,9pt]{article} % Default font size and paper size
%\usepackage{fontspec} % For loading fonts
\usepackage{eurosym}
\usepackage[graphicx]{realboxes}
\usepackage{tabularx}
\usepackage[tight,footnotesize]{subfigure}
\usepackage{graphicx,multicol}
\usepackage{xspace}
\usepackage{enumitem}
%\usepackage[utf8]{vietnam}


%% IF YOU WANT TO COMPILE THIS TEX LOCALLY, COMMENT THE FOLLOWING LINES IF THE FONTS ARE NOT INSTALLED
%\defaultfontfeatures{Mapping=tex-text}
%\setmainfont{Fontin-Regular.otf}[
%BoldFont       = Fontin-Bold.otf ,
%ItalicFont     = Fontin-Italic.otf ,
%SmallCapsFont  = Fontin-SmallCaps.otf ]
%% UNTIL HERE 

%
%\usepackage[utf8]{inputenc}
%\usepackage[english]{babel}
%\usepackage{biblatex}
%\addbibresource{main}



\usepackage{url,parskip} % Formatting packages
%\usepackage{xunicode,xltxtra,url,parskip} % Formatting packages
\usepackage[usenames,dvipsnames]{xcolor} % Required for specifying custom colors
\usepackage[big]{layaureo} % Margin formatting of the A4 page, an alternative to layaureo can be \usepackage{fullpage}
% To reduce the height of the top margin uncomment: \addtolength{\voffset}{-1.3cm}
\usepackage{hyperref} % Required for adding links	and customizing them
\definecolor{linkcolour}{rgb}{0,0.2,0.6} % Link color
\hypersetup{colorlinks,breaklinks,urlcolor=linkcolour,linkcolor=linkcolour} % Set link colors throughout the document
\usepackage{titlesec} % Used to customize the \section command
\titleformat{\section}{\Large\scshape\raggedright}{}{0em}{}[\titlerule] % Text formatting of sections
%\titleformat{\subsection}{\normalsize\scshape\raggedright}{}{0em}{}%[\titlerule] % Text formatting of sections
\titlespacing{\section}{0pt}{3pt}{3pt} % Spacing around sections

\newcommand*{\ie}{i.e.,\@\xspace}
\newcommand*{\eg}{e.g.,\@\xspace}

\renewcommand\refname{}

\hypersetup{
	colorlinks,
	citecolor=Blue,
	linkcolor=Blue,
	urlcolor=Blue}

%\usepackage[backend=bibtex,
%defernumbers=true,
%citestyle=authoryear-ibid,
%bibstyle=ieee,
%sorting=ydnt,
%backref=true,
%backrefstyle=none,
%locallabelwidth=true,
%url=false,
%firstinits=true,
%]{biblatex}

%----------------------------------------------------------------------------------------
%----------------------------------------------------------------------------------------
%	MAIN DOCUMENT - PLEASE ENTER YOUR DATA HERE
%
%	Some basic LaTeX commands: 
%
%	A '%' marks a comment that will not be shown in the final PDF
%	If you want to remove content, just comment the line(s) with a leading '%'
%
%	When you open an environment with a '{' you need to close it with a corresponding '}'!
%
%	\textbf{TEXT} 						makes TEXT bold font
%	\emph{TEXT} 						makes TEXT italics font
%	\textsc{TEXT}						makes TEXT small caps font
%	\small \footnotesize \Huge ...		defines font sizes
%	\begin{tabular} ... \end{tabular} 	defines a table environment
%										inside a table environment a '&' sign splits between columns 
%										inside a table a '\\' marks the end of a row
%----------------------------------------------------------------------------------------
%----------------------------------------------------------------------------------------
\begin{document}
%\pagestyle{empty} % Removes page numbering
%\font\fb=''[cmr10]'' % Change the font of the \LaTeX command under the skills section


%----------------------------------------------------------------------------------------
%	NAME AND CONTACT INFORMATION
%----------------------------------------------------------------------------------------
\par{\centering{\LARGE \textsc{Phuong Thanh Nguyen}}\bigskip\par} % Your name

\section{Personal information}

\begin{tabular}{rl}
\textsc{Name:} 						& Phuong Thanh NGUYEN \\%(In Vietnamese: Nguyễn Thanh Phương) \\
\textsc{Gender:}                  	& Male \\
%\textsc{Date of Birth:}             & 10$^{th}$ September 1979 \\
%\textsc{Family status:}				& Married, three children \\
%\textsc{Nationality:}               & Vietnamese\\
\textsc{Position:}               & Tenure-track Assistant Professor (RTD/b), University of L'Aquila \\
\textsc{email:} 					& \href{mailto:phuong.nguyen@univaq.it}{phuong.nguyen@univaq.it} \\ %, \href{mailto:phuong.nguyen@duytan.edu.vn}{phuong.nguyen@duytan.edu.vn}
%\textsc{Phone:} 					& +39 389 510 7723 (WhatsApp, Viber, Telegram) \\ %, \href{mailto:phuong.nguyen@duytan.edu.vn}{phuong.nguyen@duytan.edu.vn}
\end{tabular}
%----------------------------------------------------------------------------------------
%	WORK EXPERIENCE 
%----------------------------------------------------------------------------------------
%\section{Short biography}
%
%Dr. Phuong Thanh Nguyen obtained a Ph.D. in Computer Science from Friedrich-Schiller-Universit\-ät Jena (Germany). He has worked as a research and teaching assistant at various universities in Vietnam. %, including FPT University and Duy Tan University. 
%In 2014, Phuong was a postdoctoral researcher at Politecnico di Bari (Italy), working with recommender systems, Semantic Web, and Linked Data. After that, from August 2017 to January 2022, Phuong held a position as a postdoctoral researcher at University of L'Aquila %Università degli Studi dell’Aquila 
%(Italy). %Linked Open Data. %under the supervision of Prof. Dr. Tommaso Di Noia. 
%Since February 2022, he has been a tenure track assistant professor at the same university, %. Università degli Studi dell’Aquila (Italy), 
%doing research %developing techniques and tools 
%in Software Engineering, Model-Driven Engineering, and Machine Learning.
%
%His research interests include Machine Learning developments in Software Engineering and Model-Driven Engineering with applications in computer networks, semantic web, recommender systems, and classification/clustering of modeling repositories. Phuong has worked on different European projects including CROSSMINER and TYPHON, defining recommender systems to support software development and design of hybrid persistence systems.
%
%%Recently, Phuong has been working in the CROSSMINER project context to develop recommender systems for mining open source code repositories.
%


%\setcounter{tocdepth}{3}
%\setcounter{secnumdepth}{2}
%\tableofcontents


%Phuong obtained a PhD in Computer Science from the University of Jena, Germany. He has worked as a university teaching and research assistant in Vietnam and Italy. Phuong is now with the University of L'Aquila, Italy, as a postdoctoral researcher. His research interests include Computer Networks, Semantic Web, Recommender Systems, and Machine Learning. Recently, he has been working to develop recommender systems for mining open source code repositories.

%%BEGIN TABLE, one line is one row in the final table. Each row has to end with two backslash: \\
%%								columns are separated by a '&'. The table only has 2 columns! So one '&' per row
%\begin{}

%Short introduction

%\end{tabular}

%----------------------------------------------------------------------------------------
%	WORK EXPERIENCE 
%----------------------------------------------------------------------------------------
%\section{Work Experience}

%%BEGIN TABLE, one line is one row in the final table. Each row has to end with two backslash: \\
%%								columns are separated by a '&'. The table only has 2 columns! So one '&' per row
%\begin{tabular}{r|p{11cm}}
%12/2014 -- present & Researcher at Department of Nutrition for Noncommunicable Diseases,\\& National Institute of Nutrition, Vietnam Ministry of Health,\\ & Associate. Prof. Dr. National Institute of Nutrition, \\ & Vietnam Ministry of Health \\
%09/2010 -- 11/2014 & PhD student under supervision of Prof. Dr. Ivo Steinmetz,\\ & University of Greifswald, Germany\\
                
%07/2004--08/2010 & Assistant Researcher at Department of Virology, \\ & National Institute of Hygiene and Epidemiology,\\ & Vietnam Ministry of Health \\				
%\end{tabular}
%%END TABLE
%----------------------------------------------------------------------------------------
%	SCIENTIFIC EDUCATION
%----------------------------------------------------------------------------------------
\section{Education}

%%BEGIN TABLE
\begin{tabular}{r|l}	
12/2009 -- 09/2012 							& \textsc{Doctorate degree}, Dr.-Ing. \\
									& Friedrich-Schiller-Universität Jena (Germany) \\
%\footnotesize{09/2009--09/2012} 	& \footnotesize{PhD student at the University of Jena, Germany}\\
%\footnotesize{Thesis title} & \footnotesize{Characterization of putative virulence-associated genes of \textit{Burkholderia pseudomallei}}\\
%\footnotesize{Link} &\footnotesize{\url{https://epub.ub.uni-greifswald.de/files/1450/diss_Linh_Duong_Tuan_.pdf}}\\


\multicolumn{2}{c}{} \\	% this is just an empty row

09/2002 -- 02/2005 							& \textsc{Master of Information Technology}\\
          							& Hanoi University of Science and Technology (Vietnam)\\ 
          					
%\footnotesize{09/2006--12/2009} 	& \footnotesize{Studies of Microbiology at Vietnam National University, Hanoi, Vietnam}\\
%\footnotesize{Thesis title} &\footnotesize{Study some epidemiological characters of scrub typhus \textit{Rickettsia} \textit{(Orientia)} \textit{tsutsugamushi}\\& \footnotesize{isolated in Khanh Hoa province}}}\\

\multicolumn{2}{c}{} \\	% this is just an empty row

09/1997 -- 05/2002 							& \textsc{Diploma in Information Technology}\\
									& Hanoi University of Science and Technology (Vietnam)\\ 
%\footnotesize{09/1997 -- 05/2002} & \footnotesize{studied in the Joint Graduate Education Program cooperated among}\\ & \footnotesize{University of Greifswald, Germany;}\\ & \footnotesize{Vietnam National University, Hanoi, Vietnam;}\\ & \footnotesize{And Institute of Biotechnology, Vietnam Academy of Science and Technology.}
         

%\multicolumn{2}{c}{} \\	% this is just an empty row

%7/2004 							& \textsc{Bachelor of Science in Biology}\\
%          							& Vietnam National University, Hanoi, Vietnam\\ 
          					
%\footnotesize{9/2000--7/2004} 	& \footnotesize{Studies of Microbiology at University of Science, Hanoi, Vietnam}\\
%\footnotesize{Thesis title} & \footnotesize{Characterization of some biochemical immunology of \textit{Aeromonas hydrophila}} \\& \footnotesize{and usage to vaccine trial}\\
\end{tabular}

%\section{Work experience}
%%\textbf{2019 – present: Visiting Researcher at Institute of Research and Development - Duy Tan University - Vietnam}:
%
%\subsection*{Current position}
%\begin{itemize}
%	\item 02/2022 -- present: Tenure Track Assistant Professor (RTD/b), Università degli Studi dell’Aquila (Italy).	
%%	\item Sept 2005--Sept 2009: University Teacher, Duy Tan University.
%\end{itemize}
%
%\subsection*{Previous employment}
%\begin{itemize}	
%	\item 08/2017 -- 01/2022: Postdoctoral researcher, Università degli Studi dell’Aquila (Italy), working under the supervision of Prof. Dr. Davide Di Ruscio.
%	\item 06/2015 -- 07/2017: University teacher, Duy Tan University and FPT University. %(Vietnam).
%	\item 05/2014 -- 05/2015: Postdoctoral researcher, Politecnico di Bari, working under the supervision of Prof. Dr. Tommaso Di Noia. %(Italy).
%	\item 09/2005 -- 06/2008: University teacher, Duy Tan University. %(Vietnam).
%	\item 09/2002 -- 08/2005: Research and teaching assistant at Hanoi University of Science and Technology. %(Vietnam).
%\end{itemize}

\section{Habilitation}

In July 2023, I obtained the Italian habilitation (Abilitazione Scientifica Nazionale, ASN 2021 Quinto Quadrimestre) as Associate Professor for the following two independent sectors:

\begin{itemize}
	\item Computer Science (01/B1: Informatica, II fascia).\footnote{Settore Concorsuale 01/B1 - II Fascia - Quinto Quadrimestre: \url{https://bit.ly/45fVov8}}
	\item Computer Engineering (09/H1: Sistemi di Elaborazione delle Informazioni, II fascia).\footnote{Settore Concorsuale 09/H1 - II Fascia - Quinto Quadrimestre: \url{https://bit.ly/47DMZ6g}}
\end{itemize}


\section{Research interests}
\begin{itemize}
	\item \textbf{Mining Software Repositories}. Open-source software (OSS) forges, such as GitHub or Maven, offer many software projects that deliver stable and well-documented 
	products. Most OSS forges typically sustain vibrant user and expert communities which in turn provide decent support, both for 
	answering user questions and repairing reported software bugs. Moreover, OSS platforms are also an essential source of consultation for 
	developers in their daily development tasks. %Code reusing is an intrinsic feature of OSS, and developing new software by 	leveraging existing open source components allows one to considerably reduce their development effort. 
	We have conceptualized techniques and tools %~\cite{9359479},\cite{DiRoccoEMSE2020},\cite{RUBEI2020106367},\cite{10.1145/3382494.3410690} 
	to assist developers in their programming tasks.
	\item \textbf{Recommender Systems}. In online shopping platforms, recommender systems are considered to be an indispensable component, allowing business owners to offer 
	personaliz\-ed products to customers. The development of such systems has culminated in well-defined recommendation algorithms, which in turn prove their usefulness in other fields, such as entertainment industry, %~\cite{DBLP:conf/fdse/NguyenL15},\cite{10.1145/2740908.2742141},\cite{DBLP:conf/semweb/NguyenTNS15}, 
	or employment-oriented service. Recommen\-der systems in software engineering (RSSE) have been conceptualized on a comparable basis, \ie they assist developers in navigating large information spaces and getting instant recommendations that are helpful to solve a particular development task. In this sense, RSSE provide developers with useful recommendations, which may consist of different items, such as code examples, topics, %, %,\cite{DiRoccoEMSE2021}, %reusable source code, 
	third-party components, documentation, to name a few. 
	
%	\item \textbf{Machine Learning and Deep Learning}. %In recent years, 
%	The proliferation of disruptive Machine Learning (ML) and especially Deep Learning (DL) algorithms has enabled a plethora of applications across several domains. Such techniques work on the basis of complex artificial neural networks, which are capable of effectively learning from data by means of a large number of parameters distributed in different network layers.
%	%The proliferation of advanced Machine Learning algorithms enables a numerous number of applications in various domains. 
%	In this way, they are able to simulate humans' cognitive functions, aiming to acquire real-world knowledge autonomously.
%	%ML techniques are capable of conceptualizing from concrete examples, without needing to be manually coded~\cite{Domingos:2012:FUT:2347736.2347755,doi:10.1080/21693277.2016.1192517}. %Thanks to this characteristic, they have applications in various domains. 
%	ML/DL techniques are an advanced paradigm that brings in substantial improvement in performance compared to conventional learning algorithms. We have successfully studied and deployed various Machine Learning techniques in Software Engineering and other domains~\cite{NGUYEN2022117267},\cite{NguyenSoSyM2021},\cite{DUONG2023120107},\cite{9345512},\cite{NGUYEN2021110860},\cite{DUONG2020105326},\cite{8906979}.%,\cite{DuongmedRxiv}. 

%	\item \textbf{Semantic Web and Linked Data}. The natural evolution of the World Wide Web from a set of interlinked documents to a set of interlinked entities resulted in the Web of Data. It is a graph of information resources interconnected by semantic relations, thereby yielding the name Linked Data. The proliferation of Linked Data in recent year is an opportunity to create a new family of data-intensive applications such as recommender systems~\cite{DBLP:conf/fdse/NguyenL15},\cite{10.1145/2740908.2742141},\cite{DBLP:conf/semweb/NguyenTNS15}. %The deployment of recommender systems using Linked Open Data. 
%	\item \textbf{Computer Networks}. Communication is considered as a building block for mobile agent systems. In highly dynamic networks, thanks to environmental stimuli such as changes in connection quality and network topology, performance of communication among agents may be degraded considerably. Aiming to obtain fault tolerance and reliability, we proposed context-aware architectures for agent communication model inspired by the honey bee colony%To validate the hypothesis, a software prototype has been designed and implemented according to the proposed mechanism. Encouraging experimental results on a test system show that our approach brings benefits to a colony of agent platforms
%	~\cite{NguyenNICS2014},\cite{DBLP:conf/iccasa/Nguyen13}. Software-Defined Networking is a novel paradigm, based on the separation of the data plane from the control plane. It facilitates direct access to the forwarding plane of a network switch or router over the network. Though it has a lot advantages, the SDN technology leaves considerable room for improvement. Research problems like efficient techniques for customization and optimization for SDN networks are under investigation. In this way, we proposed a compact and efficient model for traffic engineering in SDN-based networks~\cite{DBLP:conf/ictcc/NguyenLZ14}.
\end{itemize}


	%ML/DL techniques 
%Thus, they have been widely applied in Web search, %by learning from a user’s long-term search history
%, %. For recommender systems, ML algorithms demonstrate their superiority by analyzing sentiment with ensemble techniques in social
%recommender systems, %or allowing systems to learn from various profiles, thus boosting up the recommendation outcomes~\cite{PORTUGAL2018205}. %Machine Learning algorithms are also indispensable to the controlling of 
%or self-driving cars, to name but a few. %In the Health care sector, the potential of ML/DL to allow for rapid diagnosis of diseases has also been proven by various research work. 
%======================================
%Deep Learning (DL) algorithms enable machines to simulate humans' learning activities, and %to autonomously learn from concrete examples. %without being constantly supervised by humans. 
%acquire real-world knowledge by generalizing from data. In this way, they %deep neural networks 
%are capable of %simulating humans' learning activities, and acquiring real-world knowledge by generalizing from data. Consequently, they can identify 
%identifying patterns and making decisions solely by means of data, without resorting to constant interventions from humans. %being manually coded.
%======================================


%Machine learning (ML) has made profound progress in the past decade, thanks to the proliferation of several disruptive deep learning algorithms. There is a rise of applications exploiting ML across several domains. Among others, ML techniques have been used to solve different issues in the agriculture sector. 

%\begin{itemize}
%    \item \textbf{Principal investigator}: %Apply deep learning for medical imaging recognition tasks which focus on classification of Lungs images related to certain diseases such as \textbf{SARS-CoV-2} and \textbf{Tuberculosis};
%%    \item \textbf{Principal investigator}: carry artificial intelligence out agriculture such as fruits and vegetables classification, and plant diseases recognition;
%%    \item \textbf{Participant} in Project: to deploy research results into mobile application and hospital computer-aid systems;
%%    \item \textbf{Participant} in Project: evaluate follow-up research of computer-aid tools for Lungs detection diseases in hospitals.
%\end{itemize}

%\textbf{2015 – present: Researcher at Department of Nutrition for Noncommunicable Diseases - National Institute of Nutrition – Vietnam Ministry of Health}:
%\begin{itemize}
%    \item \textbf{Principal investigator}: abc;
%%    \item \textbf{Principal investigator}: “Characterization of genes-associated virulent factors of some bacteria species isolated from meats sold in markets and supermarkets in Hanoi, 2018” funded by Vietnam Ministry of Health;
%%    \item \textbf{Participant} in Project “Primarily Study Gut Micriobiome of Type 2 Diabetes Mellitus Patients Living in Hanoi, Vietnam” funded by Vietnam Ministry of Health, 2020;
%%    \item \textbf{Participant} in Project “A 5-year prospective study on type 2 diabetes and metabolic syndrome in Vietnamese: role of genetic and lifestyle-related factors” sponsored by National Foundation for Science & Technology Development, Vietnam Ministry of Science and Technology;
%%    \item \textbf{Participant} in Project: “Surveillance of viral hepatitis infections in multi-geographic regions in Vietnam” granted by Vietnam-Russia Tropical Centers, Vietnam Ministry of Health;
%%    \item \textbf{Participant} in Project: “Surveillance of \textit{Orientia tsutsugamushi} infections in multi-geographic regions in Vietnam, and isolation of antigenic membrane 56 kDa for production of ELISA kit” granted by Vietnam Ministry of Health;
%%    \item \textbf{Topics of interest}: emerging and reemerging infectious diseases like diarrhea, meningitidis; zoonotic diseases such as Dengue virus, Influenza virus, \textit{Rickettsia} spp., MERS-CoV, SARS-CoV-1, SARS-CoV-2; environmental niches of microorganism pathogens; virulence factors; microbiome; host-pathogen interaction; bioinformatics; human genetics and lifestyle related to non-communicable diseases; machine learning; and Deep Learning for medical image recognition.
%\end{itemize}

%\textbf{2009 – 2014: PhD student at Friedrich Loeffler Institute of Medical Microbiology, Greifswald University Hospital, University of Greifswald, Germany }:
%\begin{itemize}
%    \item abc.
%\end{itemize}









%\section{Teaching}
%
%%\begin{}
%\subsection*{Academic year 2016 -- 2017}
%
%\begin{itemize}
%	\item Management Information Systems: FPT University, Master degree. 
%	\item Network Programming: FPT University, Bachelor degree. 
%	\item Network Security: Duy Tan University, Bachelor degree. 
%	\item Recommender Systems: Duy Tan University, Bachelor degree. 
%\end{itemize}
%
%
%\subsection*{Academic year 2015 -- 2016}
%
%\begin{itemize}
%	\item Management Information Systems: FPT University, Master degree. 
%	\item Recommender Systems: Duy Tan University, Master degree. 
%	\item Network Security: Duy Tan University, Bachelor degree. 	
%\end{itemize}
%
%
%\subsection*{Academic year 2013 -- 2014}
%
%\begin{itemize}
%	\item Management Information Systems: FPT University, Master degree. 
%	\item Recommender Systems: Duy Tan University, Master degree. 
%	\item Network Security: Duy Tan University, Bachelor degree. 	
%\end{itemize}
%
%\subsection*{Academic year 2012 -- 2013}
%
%\begin{itemize}
%	\item Management Information Systems: FPT University, Master degree. 
%%	\item Recommender Systems: Duy Tan University (Vietnam), Master degree. 
%	\item Network Security: Duy Tan University, Bachelor degree. 	
%\end{itemize}
%
%\subsection*{Academic year 2007 -- 2008}
%
%\begin{itemize}
%	\item Computer Networks: Duy Tan University, Bachelor degree. 
%	\item Network Programming: Duy Tan University, Bachelor degree. 
%	%	\item Software Testing: Hanoi University of Science and Technology (Vietnam), Bachelor degree. 
%	%	\item Network Security: Duy Tan University (Vietnam), Bachelor degree. 	
%\end{itemize}
%
%\subsection*{Academic year 2006 -- 2007}
%
%\begin{itemize}
%	\item Computer Networks: Duy Tan University, Bachelor degree. 
%	\item Network Programming: Duy Tan University, Bachelor degree. 
%	%	\item Software Testing: Hanoi University of Science and Technology (Vietnam), Bachelor degree. 
%	%	\item Network Security: Duy Tan University (Vietnam), Bachelor degree. 	
%\end{itemize}
%
%\subsection*{Academic year 2005 -- 2006}
%
%\begin{itemize}
%	\item Network Programming: Hanoi University of Science and Technology, Bachelor degree. 
%	\item Software Testing: Hanoi University of Science and Technology, Bachelor degree. 
%%	\item Network Security: Duy Tan University (Vietnam), Bachelor degree. 	
%\end{itemize}




%\begin{table}[h!]
%	\small
%	\centering
%	\caption{Bachelor courses.} 
%	\begin{tabular}{|p{5.0cm}| p{3.0cm}|p{4.5cm}|} \hline
%		\textbf{Name} & \textbf{Duration}  &  \textbf{Institution}    \\ \hline        
%		Network Programming   &  04/2016 -- 06/2016 &  FPT University (Vietnam)  \\ \hline	
%		Network Security   & 09/2016 -- 11/2016 & Duy Tan University (Vietnam) \\ \hline
%		Recommender Systems   &  04/2017 -- 06/2017 &  FPT University (Vietnam)  \\ \hline					
%	\end{tabular}
%	\label{tab:Bachelor}
%\end{table}


%\begin{table}[h!]
%	\small
%	\centering
%	\caption{Master courses.} 
%	\begin{tabular}{|p{5.0cm}| p{3.0cm}|p{4.5cm}|} \hline
%		\textbf{Name} & \textbf{Duration}  &  \textbf{Institution}    \\ \hline        
%		Scientific Writing    &  11/2016 -- 02/2017 &  FPT University (Vietnam)  \\ \hline					
%		Recommender Systems   &  04/2017 -- 06/2017 &  FPT University (Vietnam)  \\ \hline					
%		Network Security   & 09/2016 -- 11/2016 & Duy Tan University (Vietnam) \\ \hline
%		Management Information Systems   &  04/2016 -- 06/2016 &  FPT University (Vietnam)  \\ \hline	
%	\end{tabular}
%	\label{tab:Master}
%\end{table}





%\vspace{-.4cm}
%\begin{itemize}
%	\item Network Security, Winter semester 2016, Duy Tan University (Vietnam).
%	\item Management Information Systems, Winter semester 2016, FPT University (Vietnam).
%\end{itemize}

%\vspace{-.4cm}

%\subsection*{Bachelor courses}
%\vspace{-.4cm}
%\begin{itemize}
%	\item Computer Networks, Winter semester 2005, 2006, FPT University (Vietnam).
%	\item Network Security, Winter semester 2005, 2006, FPT University (Vietnam).
%	\item Recommender systems, Summer semester 2015, 2016, FPT University (Vietnam).
%\end{itemize}



%
%\section{Supervision}
%
%
%\subsection*{Doctoral theses}
%\begin{itemize}
%	\item Co-supervisor for Riccardo Rubei, University of L'Aquila (graduated with distinction).
%	\item Co-supervisor for Claudio Di Sipio, University of L'Aquila (graduated with distinction).
%\end{itemize}
%
%\subsection*{Master theses}
%
%\begin{itemize}
%	\item Aswin Palathumveettil Jagadeesan, University of L'Aquila, academic year 2022 -- 2023.
%	\item Moldir Koishybayeva, University of L'Aquila, academic year 2022 -- 2023.
%	\item Michael Dubem Igbomezie, University of L'Aquila, academic year 2022 -- 2023.
%	\item Daria Butyrskaya, University of L'Aquila, academic year 2022 -- 2023.
%	\item Farkhad Kuanyshkereyev, University of L'Aquila, academic year 2022 -- 2023.
%	\item Ashish Dahal, University of L'Aquila, academic year 2022 -- 2023.
%	\item Riccardo Rubei, University of L'Aquila, academic year 2018 -- 2019.
%	\item Thao Nguyen, FPT University (Vietnam), academic year 2016 -- 2017. %``\emph{Management Information Systems}'' 
%	\item Ngoc Nguyen, FPT University (Vietnam), academic year 2016 -- 2017.
%	\item Van Tran, FPT University (Vietnam), academic year 2016 -- 2017.
%%	\item Thuy Nguyen, Duy Tan University, bachelor thesis, academic year 2016 -- 2017.
%%	\item Huy Thai, Duy Tan University, bachelor thesis, academic year 2016 -- 2017.
%%	\item Tuan Ngo, Duy Tan University, bachelor thesis, academic year 2016 -- 2017
%%	\item Anh Luong, Duy Tan University, bachelor thesis, academic year 2015 -- 2016.
%%	\item Duyen Nguyen, Duy Tan University, bachelor thesis, academic year 2015 -- 2016.
%%	%	\item Duyen Nguyen, FPT University, bachelor thesis, academic year 2015 -- 2016.
%%	\item Long Hoang,  Duy Tan University, bachelor thesis, academic year 2015 -- 2016. %Sebastian Schaff (University of Jena, Germany).
%%	%	\item Co-supervisor for PhD students: Riccardo Rubei, Claudio Di Sipio, University of L'Aquila (Italy), ongoing.
%\end{itemize}
%
%
%%\subsection*{Ongoing theses}
%%
%%\begin{itemize}
%%	\item Co-supervisor for PhD thesis: Claudio Di Sipio, University of L'Aquila (Italy).
%%	%	\item Co-supervisor for PhD thesis: Riccardo Rubei, University of L'Aquila (Italy).
%%	%	\item Co-supervisor for PhD students: Riccardo Rubei, Claudio Di Sipio, University of L'Aquila (Italy), ongoing.
%%\end{itemize}
%
%
%\subsection*{Mentoring}
%
%
%\begin{itemize}
%	\item Since 2017, Phuong has been supervising a small group of bachelor and master students in Vietnam to work with Machine Learning and to become familiar with research activities. He has held a series of tutorials (both offline and online) about Machine Learning and scientific writing to train and inspire the group. Recently, the students started to work more with Deep Learning and write research papers, some of which have been accepted to various  journals~\cite{DuongSOCO2023},\cite{DUONG2023120107},\cite{DuongASOC2023},\cite{DuongASOC2022},\cite{DuongESWA2021},\cite{DUONG2020105326}. %,\cite{DuongCOMPAG2021}. %,\cite{DuongNCAA2021}. 
%	%One of the papers has been accepted for publication~\cite{DUONG2020105326}.
%	%	\item \emph{Linked Open Data as input for recommender systems}, Duy Tan University (Vietnam), February 2017.
%	%	\item \emph{Recommender systems}, FPT Software Company (Vietnam), October 2016.
%	%	\item \emph{Machine learning: Concepts and Applications}, FPT Software Company, April 2016.
%\end{itemize}
%




%
%\section{Supervision}
%
%\subsection*{Completed theses}
%
%\begin{itemize}
%	\item Co-supervisor: Riccardo Rubei, Università degli studi dell'Aquila (Italy), master thesis, academic year 2017 -- 2019.
%	\item Thao Nguyen, FPT University, master thesis, academic year 2016 -- 2017. %``\emph{Management Information Systems}'' 
%	\item Ngoc Nguyen, FPT University, master thesis, academic year 2016 -- 2017.
%	\item Van Tran, FPT University, master thesis, academic year 2016 -- 2017.
%	\item Thuy Nguyen, Duy Tan University, bachelor thesis, academic year 2016 -- 2017.
%	\item Huy Thai, Duy Tan University, bachelor thesis, academic year 2016 -- 2017.
%	\item Tuan Ngo, Duy Tan University, bachelor thesis, academic year 2016 -- 2017
%	\item Anh Luong, Duy Tan University, bachelor thesis, academic year 2015 -- 2016.
%	\item Duyen Nguyen, Duy Tan University, bachelor thesis, academic year 2015 -- 2016.
%%	\item Duyen Nguyen, FPT University, bachelor thesis, academic year 2015 -- 2016.
%	\item Long Hoang,  Duy Tan University, bachelor thesis, academic year 2015 -- 2016. %Sebastian Schaff (University of Jena, Germany).
%%	\item Co-supervisor for PhD students: Riccardo Rubei, Claudio Di Sipio, University of L'Aquila (Italy), ongoing.
%\end{itemize}
%
%
%\subsection*{Ongoing theses}
%
%\begin{itemize}
%	\item Co-supervisor for PhD thesis: Claudio Di Sipio, University of L'Aquila (Italy).
%	\item Co-supervisor for PhD thesis: Riccardo Rubei, University of L'Aquila (Italy).
%%	\item Co-supervisor for PhD students: Riccardo Rubei, Claudio Di Sipio, University of L'Aquila (Italy), ongoing.
%\end{itemize}


%\subsection*{Mentoring}
%
%
%\begin{itemize}
%	\item Since 2017, Phuong has been supervising a small group of bachelor and master students in Vietnam to work with Machine Learning and to become familiar with research activities. He has held a series of tutorials (both offline and online) about Machine Learning and scientific writing to train and inspire the group. Recently, the students started to work more with Deep Learning and write research papers submitted to various scientific journals~\cite{DuongESWA2021},\cite{DUONG2020105326}.%,\cite{DuongASOC2021}. %,\cite{DuongCOMPAG2021}. %,\cite{DuongNCAA2021}. 
%	%One of the papers has been accepted for publication~\cite{DUONG2020105326}.
%	%	\item \emph{Linked Open Data as input for recommender systems}, Duy Tan University (Vietnam), February 2017.
%	%	\item \emph{Recommender systems}, FPT Software Company (Vietnam), October 2016.
%	%	\item \emph{Machine learning: Concepts and Applications}, FPT Software Company, April 2016.
%\end{itemize}




%\subsection*{Manuscripts and non peer-reviewed preprints}

%%\end{}
%\section{Referees}
%\begin{itemize}
%    \item Prof. Dr. Davide Di Ruscio\\
%%	National Institute of Nutrition – Vietnam Ministry of Health\\
%%	48B Tang Bat Ho Street – Hai Ba Trung District, Hanoi, Vietnam\\
%%	Email: \href{mailto:tranquangbinh@dinhduong.org.vn}{tranquangbinh@dinhduong.org.vn}
%	%Tel: + 84904470844
%
%    \item Dr.-Ing. Phuong T. Nguyen\\
%	University of L’Aquila\\
%    Department of Information Engineering, Computer Science and Mathematics\\
%    Address: Via Vetoio, 67100, L'Aquila, Italia\\
%    Email: \href{mailto:phuong.nguyen@univaq.it}{phuong.nguyen@univaq.it} \\
%\end{itemize}

%\section{Review experience}

%\subsection*{Journals}
%\vspace{-.2cm}
%\begin{itemize}
%	\item ACM Transactions on Software Engineering and Methodology (TOSEM).
%	\item AIM Press Mathematical Biosciences and Engineering (MBE).
%	\item Elsevier Computers in Biology and Medicine (CIBM).
%	\item Elsevier Electronic Commerce Research and Applications (ECRA).
%	\item Elsevier Expert Systems with Applications (ESWA).	
%	\item Elsevier Information Processing in Agriculture (IPA).
%	\item Elsevier Journal of Systems and Software (JSS).
%	\item Elsevier Medical Image Analysis (MedIA).
%	\item Elsevier Science of Computer Programming (SCICO).
%	\item IEEE Transactions on Software Engineering (TSE).
%	\item IET Communications.
%	\item IET Software.
%	\item IOP Physics in Medicine and Biology.
%	\item Journal of Circuits, Systems, and Computers (JCSC). 
%%	\item Journal of Tuberculosis Research.
%%	\item MDPI Algorithms.
%%	\item MDPI Applied Sciences.
%%	\item MDPI Electronics.
%%	\item MDPI Journal of Imaging.
%%	\item MDPI Mathematics.
%%	\item MDPI Sensors.
%%	\item MDPI Machines.
%	\item PeerJ Computer Science. 
%	\item Springer Artificial Intelligence Review (AIRE).
%	\item Springer Empirical Software Engineering (EMSE).
%	\item Springer Precision Agriculture (PRAG).
%	\item Springer Soft Computing (SOCO).	
%	\item Springer Software and Systems Modeling (SoSyM).	
%	\item Springer Software Quality Journal (SQJO).	
%	\item Taylor Francis Applied Artificial Intelligence (UAAI).
%	\item Tsinghua Science and Technology (TST).
%\end{itemize}

%\vspace{-.4cm}

%\subsection*{Conferences \& Workshops}
%%\vspace{-.2cm}
%
%\textbf{2024}
%\begin{itemize}
%	\item The Research Papers Track of the 21st Joint meeting of the European Software Engineering Conference and the ACM SIGSOFT Symposium on the Foundations of Software Engineering (FSE 2024).
%	\item The Research Papers Track of the 28th International Conference on Evaluation and Assessment in Software Engineering (EASE 2024).%, 2024 	
%	\item The Research Papers Track of the 31st IEEE International Conference on Software Analysis, Evolution, and Reengineering (SANER 2024).
%	\item The Special Track on AI for Social Impact (AISI), the Thirty-Eighth AAAI Conference on Artificial Intelligence (AAAI 2024).
%	\item The Research Papers Track of the 40th IEEE International Conference on Software Maintenance and Evolution (ICSME 2024).% https://conf.researchr.org/track/icsme-2024/icsme-2024-papers
%	\item The Artifact Evaluation track of the 46th International Conference on Software Engineering (ICSE 2024).
%\end{itemize}
%
%\textbf{2023}
%\begin{itemize}
%	\item The Research Papers Track of the 20th Joint meeting of the European Software Engineering Conference and the ACM SIGSOFT Symposium on the Foundations of Software Engineering (ESEC/FSE 2023).
%	\item The Special Track on AI for Social Impact (AISI), the Thirty-Seventh AAAI Conference on Artificial Intelligence (AAAI 2023).
%	\item The Technical Papers track of the International Symposium on Empirical Software Engineering and Measurement (ESEM 2023).
%	\item The Research track of the 36th IEEE International Conference on Software Maintenance and Evolution (ICSME 2023).
%	\item The Technical Papers track of the 20th International Conference on Mining Software Repositories (MSR 2023).
%	\item The 2nd International Conference on AI Software Engineering, Software Engineering for AI (CAIN 2023, co-located with ICSE 2023).
%	\item The Poster track of the 45th International Conference on Software Engineering (ICSE 2023).
%	\item The Artifact Evaluation track of the 45th International Conference on Software Engineering (ICSE 2023).
%	\item The Technical Papers track of the ACM/IEEE International Conference on Technical Debt 2023.
%\end{itemize}
%
%\textbf{2022}
%\begin{itemize}	
%	\item The Technical Papers track of the 15th International Symposium on Empirical Software Engineering and Measurement (ESEM 2022).
%	\item The Technical Papers track of the 19th International Conference on Mining Software Repositories (MSR 2022).
%	\item The Vision papers and Emerging results track of the 25th International Conference on Evaluation and Assessment in Software Engineering (EASE 2022).
%	\item The New Ideas and Emerging Results (NIER) track of the 38th International Conference on Software Maintenance and Evolution (ICSME 2022).
%	\item The Special Track on AI for Social Impact (AISI), the Thirty-Sixth AAAI Conference on Artificial Intelligence	(AAAI 2022).
%	\item The Artifact Evaluation track of the 31st ACM SIGSOFT International Symposium on Software Testing and Analysis (ISSTA 2022).
%	\item The Artifact Evaluation track of the 37th IEEE/ACM International Conference on Automated Software Engineering (ASE 2022).
%	\item The 13th International Conference on Emerging Ubiquitous Systems and Pervasive Networks (EUSPN 2022).
%	\item The 17th Conference on Computer Science and Information Systems (FedCSIS 2022).
%	\item The 1st International Conference on Intelligence of Things (ICIT 2022).
%\end{itemize}
%
%\textbf{2014--2021}
%\begin{itemize}	
%	\item The 12th International Conference on Emerging Ubiquitous Systems and Pervasive Networks (EUSPN 2021).
%	\item The 24th European Conference on Artificial Intelligence (ECAI 2020).
%	\item The 11th International Conference on Emerging Ubiquitous Systems and Pervasive Networks (EUSPN 2020).% November 2-5, 2020, Madeira, Portugal
%	\item The 10th International Conference on Emerging Ubiquitous Systems and Pervasive Networks (EUSPN 2019).
%	\item The 2nd Workshop on Knowledge-aware and Conversational Recommender Systems (KaRS 2018).
%	\item The 17th International Web Engineering Conference (ICWE 2017).
%	\item The 2nd International Conference on Nature of Computation and Communication (ICTCC 2016).
%	\item The 5th International Conference on Context-Aware Systems and Applications (ICCASA 2016).
%	\item The 16th International Web Engineering Conference (ICWE 2016).
%	\item The 11th European Semantic Web Conference (ESWC 2014).		
%	\item The 4th International Conference on Context-Aware Systems and Applications (ICCASA 2015).
%	\item The 3rd International Conference on Context-Aware Systems and Applications (ICCASA 2014).
%%	\item The 2nd International Conference on Context-Aware Systems and Applications (ICCASA 2013).
%\end{itemize}
%

%\section{Participation in research projects and collaborations}

%\subsection*{Research projects}
%
%\begin{itemize}
%	\item The CROSSMINER project (2017 -- 2019): The EU CROSSMINER project\footnote{\url{https://www.crossminer.org}} aims to develop techniques and tools exploited cutting-edge information 
%	retrieval techniques, %to build recommender systems, 
%	providing software developers with practical advice	on various tasks through an Eclipse-based 
%	IDE and dedicated analytical Web-based dashboards. Based on the project's mining tools, developers can select open-source software 
%	and get real-time recommendati\-ons while working on their development tasks~\cite{9359479},\cite{DiRoccoEMSE2020},\cite{NGUYEN2020110460},\cite{DBLP:journals/sqj/NguyenRRR20}.	
%	\item The TYPHON project\footnote{\url{https://www.typhon-project.org/}} (2018 -- 2021): The aim of the project was to provide an industry-validated methodology and integrated technical offering for designing, developing, query\-ing, evolving, analyzing and monitoring architectures for scalable persistence of hybrid data (relational, graph-based, document-based, textual etc.). In the context of TYPHON, Phuong contributed the development of languages and tools for designing hybrid polystor\-es by taking into account the structure of the data and the available deployment resources. A recommender system was developed for supporting developers to select the storage technologies to be used for managing the designed conceptual entities.	
%%	\item A research project (2017 -- 2018) run by Duy Tan University, Vietnam for the development of techniques and tools to provide automated assistance to healthcare personnel in the city of Da Nang, Vietnam.\footnote{\url{http://dev.duytan.edu.vn:8088/}} Among other objectives, the project also aims to mentor researchers who are at the early stage of their career to %become familiar and 
%%	get involved in research activities.
%	\item The Biodiversity Exploratories project\footnote{\url{https://www.bexis.uni-jena.de/}} (2011 -- 2013), University of Jena (Germany) was conducted to build %conceptualize 
%	BExIS, a data repository and information exchange platform. The platform is now still operating to provide researchers with a means to exchange information and experience about biodiversity across Germany.
%%	\item A research project (2004 -- 2006) funded by the Municipal People’s Committee of Hanoi, the capital city of Vietnam to build a system for automated recognition of fingerprints.
%%	\item The SpeedUp project, University of Jena (Germany).	
%\end{itemize}


%detect tuberculosis from chest X-ray images
%
%\subsection*{Collaborations}
%
%\begin{itemize}
%	\item From 2021 -- present: Dr. Paola Inverardi, Full Professor, Rector of the Gran Sasso Science Institute (Italy).
%	\item From 2022 -- present: Dr. Tu Bao Ho, Professor Emeritus, Japan Advanced Institute of Science and Technology (Japan), and Vietnam Institute for Advanced Study in Mathematics (Vietnam).
%	\item From 2021 -- present: Dr. Francesca Arcelli Fontana, Full Professor, Università degli Studi di Milano-Bicocca (Italy).
%	\item From 2019 -- present: Dr. Massimiliano Di Penta, Full Professor, Università degli Studi del Sannio (Italy).
%	\item From 2019 -- present: Dr. Andrea Capiluppi, Associate Professor, University of Groningen (The Netherlands).
%	\item From 2019 -- present: Dr. Michele Flammini, Full Professor, Gran Sasso Science Institute (Italy).
%	\item From 2017 -- present: Dr. Thomas Degueule, Assistant Professor, Laboratoire Bordelais de Recherche en Informatique (France).
%%	\item From 2017 -- present: Dr. Dimitris Kolovos, Full Professor, University of York (Great Britain).
%	\item From 2017 -- present: Dr. Ludovico Iovino, Assistant Professor, Gran Sasso Science Institute (Italy).
%	\item From 2014 -- present: Dr. Tommaso Di Noia, Full Professor, Politecnico di Bari (Italy).
%	\item In 2017: Dr. Kai Eckert, Full Professor, Stuttgart Media University (Germany).
%	\item In 2017: Dr. Azzura Ragone, Senior Researcher, Università degli Studi di Milano-Bicocca (Italy).
%	\item In 2014: Dr. Thomas Zinner, Associate Professor, Julius-Maximilians-Universität Würzburg (Germany).
%	\item In 2011: Dr. Birgitta König-Ries, Full Professor, Friedrich-Schiller-Universität Jena (Germany).
%\end{itemize}







%\section{Talks and organizations}


%\subsection*{Invited talks}
%
%
%\begin{itemize}
%	\item \emph{Deep Learning for Software Engineering: The era of Large Language Models and ChatGPT}, University of Milano-Bicocca (Italy), 11/2023.
%	\item \emph{Applications of Machine Learning in Software Engineering}, Summer School on Software Engineering for Digital Society, Giulianova (Italy), 06/2023.\footnote{\url{https://www.se4ds.mdu.se/}}
%	\item \emph{Deep Learning in Software Engineering}, University of Engineering and Technology (Vietnam), 08/2022.
%	\item \emph{Applications of Recommender Systems and Machine Learning in Software Engineering}, AI Center, Hanoi University of Science and Technology (Vietnam), 08/2022 (\href{https://bit.ly/3tFEKDO}{https://bit.\-ly/3tFEKDO}).
%	\item \emph{Some thoughts on the application of Deep Learning in Model-Driven Engineering}, lightning talk at the 3rd Workshop on Artificial Intelligence and Model-driven Engineering (MDE Intelligence), co-located with MODELS 2021, 10/2021.
%	\item \emph{Recommender Systems and Machine Learning for Software Engineering}, Università degli Studi dell’Aquila (Italy), 03/2021.
%	\item \emph{Unsupervised learning for document clustering}, Banking Academy of Vietnam, 04/2017.
%%	\item \emph{Recommender systems with Linked Open Data}, Ha Long University (Vietnam), 02/2017.
%	\item \emph{Recommender systems}, FPT Software Company, 10/2016.
%	\item \emph{Machine learning: Concepts and Applications}, FPT Software Company, 04/2016.
%%	\item \emph{Supervised learning with neural networks}, Hanoi University of Mining and Geology, 03/2016.
%\end{itemize}
%
%
%\subsection*{Invited tutorials}
%
%\begin{itemize}
%	\item \emph{Software Testing}, FPT Software Company (Vietnam), 05/2017.
%	\item \emph{Software Testing}, VASC Software Company (Vietnam), 12/2016.
%\end{itemize}



%
%\subsection*{Presentations at conferences/workshops}
%
%\begin{itemize}
%	
%	\item \emph{Adversarial Machine Learning: On the Resilience of Third-party Library Recommender Systems}, virtual presentation at the 25th International Conference on Evaluation and Assessment in Software Engineering, EASE 2021, Trondheim, Norway (Virtual).%DOI: \href{https://doi.org/10.1145/3463274.3463809}{https://doi.org/10.1145/3463274.3463809}.
%	\item \emph{Building Information Systems Using Collaborative-Filtering Recommendation Techniques}, presentation at the Advanced Information Systems Engineering Workshops (co-located with CAiSE 2019), Rome, Italy.
%	\item \emph{Knowledge-aware Recommender System for Software Development}, virtual presentation at the 2nd Knowledge-aware and Conversational Recommender Systems Workshop, KaRS 2018 (co-located with RecSys 2018), October 7, 2018, Vancouver, Canada.
%	\item \emph{Mining Software Repositories to Support OSS Developers: A Recommender Systems Approach}, presentation at the 9th Italian Information Retrieval Workshop, IIR 2018, Rome, Italy.
%	\item \emph{Finding Similar Artists from the Web of Data: A PageRank Based Semantic Similarity Metric}, presentation at the 2nd International Conference on Future Data and Security Engineering, FDSE 2015, Sai Gon, Vietnam.	
%	\item \emph{A Context-Aware Traffic Engineering Model for Software-Defined Networks}, presentation at the 2nd International Conference on Nature of Computation and Communication, ICTCC 2014, Sai Gon, Vietnam.
%	\item \emph{A Context-Aware Model for the Management of Agent Platforms in Dynamic Networks}, presentation at the 2nd International Conference on Context-Aware Systems and Applica\-tions, ICCASA 2013, Phu Quoc, Vietnam.
%	\item \emph{Building Consensus in Context-Aware Systems Using Ben-Or’s Algorithm: Some Proposals for Improving the Convergence Speed}, presentation at the International Conference on Context-Aware Systems and Applications, ICCASA 2013, Phu Quoc, Vietnam.
%	\item \emph{An Adaptive Communication Model for Mobile Agents inspired by the Honey Bee Colony: Theory and Evaluation}, presentation at the 10th European Workshop on Multiagent Systems, EUMAS 2012, Maastricht, the Netherlands.
%	\item \emph{Performance comparison of some message transport protocol implementations for agent community communication}, presentation at the 11th International Conference on Innova\-tive Internet Community Systems, I2CS 2011, Berlin, Germany.
%	\item \emph{Performance Evaluation of Video SSIM Quality Metric on VQEG FR-TV Phase I Test Dataset}, presentation at the 13th International Student Conference on Electrical Enginee\-ring, POSTER 2009, Prague, Czech Republic.
%\end{itemize}


%We invite researchers and practitioners to present their
%

%\subsection*{Organization of Workshops}
%
%
%\begin{itemize}
%	
%	\item Lola Burgueño, Dominik Bork, \underline{Phuong T. Nguyen}, Steffen Zschaler, \emph{the Fourth Workshop on Artificial Intelligence and Model-driven Engineering}, MDE Intelligence 2022, co-located with MODELS 2022.\footnote{\url{https://mde-intelligence.github.io/}}	
%	\item Massimiliano Di Penta, Juri Di Rocco, \underline{Phuong T. Nguyen}, \emph{the First International Works\-hop on Evaluation and Analysis of Recommender Systems in Software Engineering} (WEA\-RS 2021).\footnote{\url{https://wears21.github.io/}} The workshop aims to bring in a forum for researchers and practitioners to share, discuss and explore the opportunities and challenges raised by the evaluation and in-depth investigation of recommender systems in Software Engineering. We solicit research work to increase the synergy among various communities, including Software Engineering, Machine Learning, and Recommender Systems.
%\end{itemize}


\section*{Editorial activities}

\begin{itemize}
	\item Associate Editor of Springer Applied Intelligence (\url{https://www.springer.com/journal/10489/editors}). 
	\item Member of the Editorial Board of the Software Quality Journal (\url{https://www.springer.com/journal/11219/editors}).
	\item Member of the Editorial Board of the Journal of Universal Computer Science (\url{https://bit.ly/3RFhtvB}).
	\item Member of the Editorial Board of Elsevier	Computers \& Education: Artificial Intelligence (\url{https://bit.ly/3fMckVi}).
%	\item Topic Editor of MDPI Informatics (\url{https://www.mdpi.com/journal/informatics}).	
%	\item Guest editor for Special Issue ``\emph{Applications of Machine Learning and Deep Learning in Agriculture},'' MDPI Informatics (\url{https://www.mdpi.com/journal/informatics/special_issues/AMLDLA}).
\end{itemize}



\section{Honours and Awards}

\subsection*{Awards}

\begin{itemize}
	
	
	\item ``\textbf{\emph{2022 SoSyM First Paper Award}}'': Juri Di Rocco, Davide Di Ruscio, Claudio Di Sipio, \underline{Phuong T. Nguyen}, Alfonso Pierantonio, ``\emph{MemoRec: A Recommender System for Assisting Modelers in Specifying Metamodels},'' Springer Software and Systems Modeling (SoSyM), DOI: \href{https://doi.org/10.1007/s10270-022-00994-2}{10.1007/s10270-022-00994-2}.
		
    \item ``\textbf{\emph{Best Foundation Paper Award}}'': Juri Di Rocco, Claudio Di Sipio, Davide Di Ruscio, \underline{Phuong T. Nguyen}, ``\emph{A GNN-based Recommender System to Assist the Specification of Metamodels and Models},'' DOI: \href{https://doi.org/10.1109/MODELS50736.2021.00016}{10.1109/MODELS50736.2021.00016} awar\-ded by the Program Board of the 24th ACM/IEEE International Conference on Model Driven Engineering Languages and Systems, MODELS 2021. %, (\href{https://bit.ly/2XeRAwi}{https://https://bit\-.ly/2XeRAwi}).
	
	\item ``\textbf{\emph{Best Paper Award Winners for 2020}}''% and ``\textbf{\emph{Diamond Best Paper Award}}''%for the following paper
	: \underline{Phuong T. Nguyen}, Juri Di Rocco, Davide Di Ruscio, Massimiliano Di Penta ``\emph{CrossRec: Supporting Software Developers by Recommending Third-party Libraries},'' Elsevier Journ\-al of Systems and Software, 2020, ISSN: 0164-1212, DOI: \href{https://doi.org/10.1016/j.jss.2019.110460}{10.1016/j.jss.2019.110460}, (\href{https://bit.ly/3bZi5cx}{https:\-//bit.ly/3bZi5cx}). %https://bit.ly/3bWDxyQ
	
	\item ``\textbf{\emph{Diamond Best Paper Award}}''%for the following paper
	: \underline{Phuong T. Nguyen}, Juri Di Rocco, Davide Di Ruscio, Massimiliano Di Penta ``\emph{CrossRec: Supporting Software Developers by Recommending Third-party Libraries},'' Elsevier Journ\-al of Systems and Software, 2020, ISSN: 0164-1212, DOI: \href{https://doi.org/10.1016/j.jss.2019.110460}{10.1016/j.jss.2019.110460}, (\href{https://bit.ly/3bZi5cx}{https:\-//bit.ly/3bZi5cx}). %https://bit.ly/3bWDxyQ

    \item ``\textbf{\emph{Best Paper Award}}'': \underline{Phuong T. Nguyen}, Juri Di Rocco, Davide Di Ruscio, Alfonso Pierantonio, Ludovico Iovino, ``\emph{Automated Classification of Metamodel Repositories: A Machine Learning Approach},'' DOI: \href{https://doi.org/10.1109/MODELS.2019.00011}{10.1109/MODELS.2019.00011}, awar\-ded by the Program Board of the 22nd ACM/IEEE International Conference on Model Driven Engineering Languages and Systems, MODELS 2019.
    
    \item ``\textbf{\emph{Distinguished paper}}'': \underline{Phuong T. Nguyen}, Juri Di Rocco, Riccardo Rubei, Davide Di Ruscio, ``\emph{CrossSim: exploiting mutual relationships to detect similar OSS projects},'' in Proceedings of the 44th Euromicro Conference on	Software Engineering and Advanced Applications, SEAA 2018, DOI: \href{https://doi.org/10.1109/SEAA.2018.00069}{10.1109/SEAA.2018.00069}, (\href{https://bit.ly/3hrPMr1}{https://bit.\-ly/3hrPMr1}).		
    
	\item ``\textbf{\emph{Best Paper Award}}'': \underline{Phuong T. Nguyen}, Hong Anh Le, Thomas Zinner ``\emph{A Context-Aware Traffic Engineering Model for Software-Defined Networks},'' DOI: \href{https://doi.org/10.1007/978-3-319-15392-6\_8}{10.1007/978-3-319-15392-6\_8}, awarded by the Program Board of the 2nd International Conference on Nature of Computation and Communication, ICTCC 2014.    
	
%    \item Annual scholarship from Hanoi University of Science and Technology (1997 -- 2002).
\end{itemize}

%\vspace{-.2cm}
%\subsection*{Scholarships}
%\begin{itemize}
%	\item Fellowship granted by Vietnam Institute for Advanced Study in Mathematics for a research stay (08/2022).
%	\item Scholarship granted by the German Academic Exchange Service (DAAD) for the PhD study in Germany (2009 -- 2012).
%	\item Full scholarship granted by the Vietnamese government for the PhD study in Germany (2009 -- 2012).
%\end{itemize}

%\vspace{-.2cm}
%\subsection*{Recognitions} % for research activities
%
%\begin{itemize}
%	\item Certificate of service from MPDI Journal of Informatics as Guest Editor of Special Issue ``\emph{Applications of Machine Learning and Deep Learning in Agriculture}.''
%	\item Reviewer Confirmation Certificate from MPDI for reviewing for Journal of Imaging, and MPDI Electronics.
%	\item Reviewer Recognition from Springer Journal of Software and Systems Modeling (SoSyM). The editors in chief issued a certificate to express their appreciation for the contributions as a SoSyM reviewer in 2020. %, from 01/01/2020 al 31-12-2017 .
%	\item Certificate of Reviewing from Elsevier Expert Systems With Applications (ESWA). A certificate issued by the editors in chief to express their appreciation for the contributions as an ESWA reviewer in 2020 and 2021.
%	\item Certificate of Reviewing from Elsevier Information Processing in Agriculture (IPA). A certificate issued by the editors in chief to express their appreciation for the contributions as an IPA reviewer in 2020.
%	\item Reviewer Certificate of Recognition from Springer Empirical Software Engineering (EMSE). The editors in chief issued a certificate to express their appreciation for the contributions as an EMSE reviewer in 2019.
%\end{itemize}




%\section{Programming and relevant skills}
%%\begin{}
%\begin{itemize}		
%	\item Programming languages: Java, Python, C/C+, C\#, ASP.
%	\item Scripting languages: HTML, \LaTeX.
%	\item Version control: Git.
%	\item IDE: Eclipse, Microsoft Visual Studio.
%	\item Data analysis/storytelling with R, Python, Open Office, \LaTeX.	
%%	\item Data visualization with R, Python, Open Office, \LaTeX.
%	\item Machine Learning frameworks/platforms: Keras, TensorFlow, Scikit-learn, Google Colabor\-atory. %Good experience with different 
%\end{itemize}
%%\end{}
%
%\section{Languages}
%
%\begin{itemize}
%    \item Vietnamese (mother tongue).
%    \item English (good).
%    \item German (B2+, good): Phuong has worked as a freelance translator in German and Vietnamese.% as a hobby.
%    \item Italian (B2, basic knowledge).
%%    \item \textbf{Soft skills}: posters and presentations, time management, leadership; teamwork, adaptability;
%%    \item \textbf{Hobbies}: playing various sports, reading, documentary films, history, and philosophy.
%\end{itemize}

%for mining open source software repositories



%\section{Future research}
%
%In the incoming years, Phuong's research is going to be focused on the development of solutions to ongoing issues in Software Engineering, including Model-Driven Engineering (MDE) based on Machine Learning/Deep Learning. In particular, there are the following core research themes:
%
%\begin{itemize}
%	\item \textbf{Mining sequential and time-series data in Software Engineering.} In open source software repositories, there exist time-series artifacts which are the result of the interaction between developers and hosting platforms, \eg the evolution of a software project in GitHub over the course of time. %contains sequential information. 
%%	or the inclusion and removal of third-party libraries over the course of time.
%	%resolving of issues. 
%	Similarly in MDE, models evolve during their lifecycle, resulting in the transformation and evolution of models. Such type of data, once being properly mined, can provide developers/modelers with useful recommendations, helping them complete their tasks.  In this respect, we assume that the deployment of Machine Learning techniques such as Long Short-Term Memory recurrent neural networks (LSTM), or Encoder-Decoder LSTM allows us to mine the existing data, providing supports to developers. As preliminary work, we proposed a novel approach to recommendation of an upgrade plan for software projects with respect to library usage~\cite{NGUYEN2022117267}. Migration history of projects is mined to build matrices, and train deep neural networks, 
%%	which have been specifically designed to cope with sequential data. 
%	whose weights and biases are used to forecast the next versions of the related libraries.
%	\item \textbf{Protecting users' privacy}. The increasingly use of distributed data to train Machine Learning models triggers concern over users' privacy. While it is necessary to improve the learning capability of such models by feeding them with more data, it is also important to preserve users' privacy. Recently, there have been studies conducted to conceive mechanisms to govern the creation, destruction, use, and sharing of data according to the owner’s ethical preferences. Among others, Federated Machine Learning (FML) has been conceived as an effective means to allow devices to learn a common prediction model, while keeping private data on their own, without needing to upload it to cloud. This allows users to prevent behaviours that are not admissible by the ethical preferences. Moreover, an important and meaningful research topic in Software Engineering is to study models that can be used to recommend \emph{privacy personas} according to users' preferences. This will enable users to automatically adapt their privacy settings when they go online.
%	\item \textbf{Adversarial Machine Learning}. Recommender systems in software engineering (RSSE) provide %developers with 
%	a wide range of useful items to assist developers in completing their programming tasks. %Among others, API recommender systems have gained momentum in recent years as they become more successful at recommending API calls and code snippets. 
%	%However, while effort has been made to make these systems become effective with respect to prediction accuracy, researchers pay less attention to their safeness/resilience, \ie protecting them against adversarial attempts. %In recent years, there is a dramatic increase in the application of Machine Learning (ML) algorithms in several domains, including the development of recommender systems for software engineering (RSSE). 
%	While research has been conducted to improve recommendation accuracy, little attention has been paid to make such systems robust and resilient to malicious data. In fact, by manipulating the algorithms' training set, \ie large open source software (OSS) repositories, a hostile user could possibly render RSSE vulnerable to adversarial attacks. Such activities generate perturbations to deceive and disrupt systems by causing a malfunction, compromising recommendation capabilities. Though the topic of Adversarial Machine Learning (AML) has been studied in a wide range of domains, \eg online shopping systems or image classification, %and addresses both risks and countermeasures. Remarkably, 
%	there has been no work dealing with adversarial attempts to RSSE~\cite{NguyenASE2021},\cite{NguyenEASE2021}. %recommender systems.
%	%	To defend RSSE against threats, %in the first place, 
%	%	it is necessary to be knowledgeable of various types of adversary activities. 
%	In this respect, there is an urgent need to explore AML in the context of Software Engineering, with the aim of conceiving mechanisms to defend recommender systems against malicious intents disguised in training data. % and Model-Driven Engineering.
%	\item \textbf{Perceiving and dealing with threats caused by GANs}. Generative Adversarial Neural Networks (GANs) are a special type of deep neural networks that have been widely used in image processing to produce crafted images, which resemble real ones. %Fake images can be used to fool Deep Learning systems. %Due to this characteristics, they are considered to be among threats for Deep Learning systems. %are a very popular class of deep generative models which have shown some incredible results in generating images. 
%%	According to my investigation, %to fool recommender systems. 
%	In a similar manner, GANs can also be exploited to generate fake training data to feed RSSE, posing a potential danger to software systems. Nevertheless, the application of GANs in Software Engineering remains unexplored. Therefore, it is crucial to investigate the possible implication of such techniques on RSSE, as well as to conceive effective countermeasures. 
%	\item \textbf{Deep Learning for the detection of technical debt, code smells}. In software development, it is important to maintain a healthy ecosystem, where software systems are expected to be free from bugs and errors. Among others, it is necessary to detect probable technical debt and code smells~\cite{DiSalleTechDebt2022}, so as to guide the maintenance process. Deep Learning techniques are supposed to bring substantial benefit in this task as they %. In recent years, the proliferation of Deep Learning (DL) algorithms has brought unprecedented benefits, paving the way for promising results and performance in different application domains~\cite{10.5555/3086952}. DL systems 
%	work by simulating the learning activities of humans, %~\cite{DBLP:journals/corr/PortugalAC15a}, 
%	being capable of autonomously learning meaningful patterns from real-world examples. %~\cite{Domingos:2012:FUT:2347736.2347755,doi:10.1080/21693277.2016.1192517}. 
%	In particular, they effectively extract the containing knowledge from data, leading to a better prediction/classification than conventional approaches.	
%\end{itemize}


%This section briefly recalls the main technical details of GANs.%\footnote{The background of GANs as well as the corresponding notions originate from \url{https://bit.ly/3hizVtf}}
%Allowing devices to learn a common prediction model, while keeping private data on their own, without uploading to cloud. 
%Aiming to preserve privacy for each participating platform. 
%Preserving privacy for the participating platforms. 

%\clearpage

%\section{Profiles}
%
%\vspace{.2cm}
%
%\begin{itemize}
%	\item DBLP: \small	\url{https://dblp.org/pid/178/5921.html}. \normalsize
%	\item Google Scholar: \small \url{https://scholar.google.com/citations?user=vxFDCLUAAAAJ\&hl=en} \normalsize
%%	\item Publons: \small \url{https://publons.com/researcher/3807477/phuong-nguyen}. \normalsize		
%%	\item ResearchGate: \small	\url{https://www.researchgate.net/profile/Phuong-Nguyen-51}. \normalsize
%	\item ORCID: \small \url{https://orcid.org/0000-0002-3666-4162} \normalsize
%	\item Scopus: \small \url{https://www.scopus.com/authid/detail.uri?authorId=57209915714}. \normalsize
%	\item Web of Science: \small \url{https://www.webofscience.com/wos/author/record/ABE-3890-2021}. \normalsize	%\url{https://www.webofscience.com/wos/author/record/2423937}
%\end{itemize}



%\smallskip

%
%\section{Publications}  
%%\begin{}
%%\subsection*{Journal papers}
%
%\begin{itemize}
%	\item Juri Di Rocco, Davide Di Ruscio, Claudio Di Sipio, \underline{Phuong T. Nguyen}, Riccardo Rubei, ``\emph{Development of recommendation systems for software engineering: the	CROSSMINER experience},'' Springer Empirical Software Engineering (EMSE), \emph{to appear}.	
%	\item \underline{Phuong T. Nguyen}, Juri Di Rocco, Claudio Di Sipio, Davide Di Ruscio, Massimiliano Di Penta ``\emph{Recommending API Function Calls and Code Snippets to Support Software Development},'' IEEE Transactions on Software Engineering (TSE), 2021, ISSN: 1939-3520, DOI: \href{https://doi.org/10.1109/TSE.2021.3059907}{10.1109/TSE\-.2021.3059907}.	
%	\item Ludovio Iovino, \underline{Phuong T. Nguyen}, Amleto Di Salle, Francesco Gallo, Michele Flammini ``\emph{Unavailable Transit Feed Specification: Making It Available With Recurrent Neural Networks},'' IEEE Transactions on Intelligent Transportation Systems (T-ITS), 2021, ISSN: 1558-0016, DOI: \href{https://doi.org/10.1109/TITS.2021.3053373}{10.1109/TITS.2021.3053373}.	
%	\item \underline{Phuong T. Nguyen}, Davide Di Ruscio, Juri Di Rocco, Ludovico Iovino, ``\emph{Convolutional neural networks for enhanced classification mechanisms of metamodels},'' Journal of Systems and Software (JSS), 2020, ISSN: 0164-1212, DOI: \href{https://doi.org/10.1016/j.jss.2020.110860}{10.1016/j.jss.2020.110860}.	
%	\item \underline{Phuong T. Nguyen}, Juri Di Rocco, Riccardo Rubei, Davide Di Ruscio, ``\emph{An Automated Approach to Assess the Similarity of GitHub Repositories},'' Software Quality Journal (SQJ), 2020, ISSN: 0963-9314, DOI: \href{https://doi.org/10.1007/s11219-019-09483-0}{10.1007/s11219-019-09483-0}.
%	%	\item Phuong T. Nguyen, Juri Di Rocco, Davide Di Ruscio, Massimiliano Di Penta, ``\emph{CrossRec: Supporting Software Developers by Recommending Third-party Libraries},'' Journal of Systems and Software,	
%	\item \underline{Phuong T. Nguyen}, Juri Di Rocco, Davide Di Ruscio, Massimiliano Di Penta, ``\emph{CrossRec: Supporting Software Developers by Recommending Third-party Libraries},'' Journal of Systems and Software (JSS), 2020, ISSN: 0164-1212, DOI: \href{https://doi.org/10.1016/j.jss.2019.110460}{10.1016/j.jss.2019.110460}.	
%	\item Andrea Capiluppi, Davide Di Ruscio, Juri Di Rocco, \underline{Phuong T. Nguyen}, Nemitari Ajienka, ``\emph{Detecting Java Software Similarities by using Different Clustering Techniques},'' Information and Software Technology (IST), 2020, ISSN: 0950-5849, DOI: \href{https://doi.org/10.1016/j.infsof.2020.106279}{10.1016/j.infsof\-.2020.106279}.	
%	\item Riccardo Rubei, Claudio Di Sipio, \underline{Phuong T. Nguyen}, Juri Di Rocco, Davide Di Ruscio, ``\emph{PostFinder: Mining Stack Overflow posts to support software developers},'' Information and Software and Technology (IST), 2020, ISSN: 0950-5849, DOI: \href{https://doi.org/10.1016/j.infsof.2020.106367}{10.1016/j.infsof\-.2020.106367}.
%	\item Linh T. Duong, \underline{Phuong T. Nguyen}, Claudio Di Sipio, and Davide Di Ruscio, ``\emph{Automated Fruits Recognition on using EfficientNet and MixNet},'' Elsevier Computers and Electronics in Agriculture, Volume 171, April 2020, 105326, \href{https://doi.org/10.1016/j.compag.2020.105326}{10.1016/j.compag.2020.105326}.
%\end{itemize}
%
%%\vspace{-.2cm}
%%\subsection*{Conference/Paper in conference proceedingss}
%%\vspace{-.2cm}
%
%\begin{itemize}	
%	\item Angela Barriga, Davide Di Ruscio, Ludovico Iovino, \underline{Phuong T. Nguyen}, Alfonso  Pierantonio ``\emph{An extensible tool-chain for analyzing datasets of metamodels},'' in Proceedings of the 23rd ACM/IEEE International Conference on Model Driven Engineering Languages and Systems: Companion Proceedings, DOI: \href{https://doi.org/10.1145/3417990.3419626}{10.1145/3417990.3419626}.
%	\item Claudio Di Sipio, Davide Di Ruscio, \underline{Phuong T. Nguyen} ``\emph{Democratizing the development of recommender systems by means of low-code platforms},'' in Proceedings of the 23rd ACM/IEEE International Conference on Model Driven Engineering Languages and Systems: Companion Proceedings, MODELS 2020, DOI: \href{https://doi.org/10.1145/3417990.3420202}{10.1145/3417990.3420202}.
%	\item \underline{Phuong T. Nguyen}, Juri Di Rocco, Davide Di Ruscio, Lina Ochoa, Thomas Degueule, Massimilian Di Penta, ``\emph{FOCUS: A Recommender System for Mining API Function Calls and Usage Patterns}.'' In Proceedings of the 41st International Conference on Software Engineering, ICSE 2019, DOI: \href{https://doi.org/10.1109/ICSE.2019.00109}{10.1109/ICSE.2019.00109}. %Lecture Notes in Computer Science, Springer. \href{\url{https://doi.org/10.1109/ICSE.2019.00109}}		
%	\item \underline{Phuong T. Nguyen}, Juri Di Rocco, Davide Di Ruscio, Alfonso Pierantonio, Ludovico Iovino, ``\emph{Automated Classification of Metamodel Repositories: A Machine Learning Approach},'' in Proceedings of the 22nd ACM/IEEE International Conference on Model Driven Engineering Languages and Systems, MODELS 2019, DOI: \href{https://doi.org/10.1109/MODELS.2019.00011}{10.1109/MODELS.20\-19.00011}. %Lecture Notes in Computer Science, Springer. 	
%	\item \underline{Phuong T. Nguyen}, Juri Di Rocco, Davide Di Ruscio: ``\emph{Enabling heterogeneous recommendations in OSS development: What's done and what's next in CROSSMINER},'' In Proceedings of the 23rd International Conference on Evaluation and Assessment on Software Engineering, EASE 2019, DOI: \href{https://doi.org/10.1145/3319008.3319353}{10.1145/3319008.33\-19353}. %Lecture Notes in Computer Science, Springer. \href{\url{https://doi.org/10.1109/ICSE.2019.00109}}	
%	\item \underline{Phuong T. Nguyen}, Juri Di Rocco, Davide Di Ruscio ``\emph{Building Information Systems Using Collaborative-Filtering Recommendation Techniques},'' in Proceedings of Advanced Information Systems Engineering Workshops, CAiSE 2019, DOI: \href{https://doi.org/10.1007/978-3-030-20948-3\_19}{10.1007/978-3-030-20948-3\_19}
%	\item \underline{Phuong T. Nguyen}, Juri Di Rocco, Riccardo Rubei, Davide Di Ruscio, ``\emph{CrossSim: exploiting mutual relationships to detect similar OSS projects},'' in Proceedings of the 44th Euromicro Conference on
%	Software Engineering and Advanced Applications, SEAA 2018, ISBN: 978-1-5386-7383-6, DOI: \href{https://doi.org/10.1109/SEAA.2018.00069}{10.1109/SEAA.2018.00069}.		
%	\item Claudio Di Sipio, Riccardo Rubei, Davide Di Ruscio, \underline{Phuong T. Nguyen}, ``\emph{A Multinomial Naïve Bayesian (MNB) Network to Automatically Recommend Topics for GitHub Repositories},'' in
%	Proceedings of the 24th International Conference on Evaluation and Assessment in Software Engineering, EASE 2020, DOI: \href{https://doi.org/10.1145/3383219.3383227}{10.1145/3383219.3383227}.	
%	\item Juri Di Rocco, Davide Di Ruscio, Claudio Di Sipio, \underline{Phuong Nguyen} and Riccardo Rubei, ``\emph{TopFilter: An Approach to Recommend Relevant GitHub Topics},'' in Proceedings of the 14th International Symposium on Empirical Software Engineering and Measurement, ESEM 2020, DOI: \href{https://doi.org/10.1145/3382494.3410690}{10.1145/3382494.341\-0690}.	
%	\item \underline{Phuong T. Nguyen}, Juri Di Rocco, Davide Di Ruscio ``\emph{Knowledge-aware Recommender System for Software Development},'' in Proceedings of the Knowledge-aware and Conversational Recommender Systems Workshop, KaRS 2018 (co-located with RecSys 2018), October 7, 2018, Vancouver, Canada, \href{http://ceur-ws.org/Vol-2290/kars2018_paper4.pdf}{(link)}.	
%	\item \underline{Phuong T. Nguyen}, Juri Di Rocco, Davide Di Ruscio ``\emph{Mining Software Repositories to Support OSS Developers: A Recommender Systems Approach},'' in Proceedings of the Italian Information Retrieval Workshop, IIR 2018 \href{http://ceur-ws.org/Vol-2140/paper9.pdf}{(link)}.
%	\item \underline{Phuong T. Nguyen}, Kai Eckert, Azzurra Ragone, Tommaso Di Noia, ``\emph{Modification to K-Medoids and CLARA for Effective Document Clustering},'' in Proceedings of the 23rd International Symposium on Methodologies for Intelligent Systems, ISMIS 2017, DOI: \href{https://doi.org/10.1007/978-3-319-60438-1\_47}{10.1007/978-3-319-60438-1\_47}. 	
%	\item \underline{Phuong T. Nguyen}, Paolo Tomeo, Tommaso Di Noia, Eugenio Di Sciascio: ``\emph{Content-based recommendations via DBpedia and Freebase: A case study in the music domain},'' in Proceedings of the 14th International Semantic Web Conference, ISWC 2015, DOI: \href{https://doi.org/10.1007/978-3-319-25007-6_35}{10.1007/978-3-319-25007-6\_35}.  	
%	\item \underline{Phuong T. Nguyen}, Paolo Tomeo, Tommaso Di Noia, Eugenio Di Sciascio: ``\emph{An evaluation of SimRank and Personalized PageRank to build a recommender system for the Web of Data},'' in Proceedings of the 7th International Workshop on Web Intelligence \& Communities, at WWW'2015, DOI: \href{https://doi.org/10.1145/2740908.2742141}{10.1145/2740908.2742141}. 	
%	\item\underline{Phuong T. Nguyen}, Hong Anh Le ``\emph{Finding Similar Artists from the Web of Data: A PageRank Based Semantic Similarity Metric},'' in Proceedings of the International Conference on Future Data and Security Engineering, FDSE 2015, DOI: \href{https://doi.org/10.1007/978-3-319-26135-5\_8}{10.1007/978-3-319-26135-5\_8}.	
%	\item \underline{Phuong T. Nguyen}, Hong Anh Le, Thomas Zinner ``\emph{A Context-Aware Traffic Engineering Model for Software-Defined Networks},'' in Proceedings of International Conference on Nature of Computation and Communication, ICTCC 2014, DOI: \href{https://doi.org/10.1007/978-3-319-15392-6_8}{10.1007/978-3-319-15392-6\_8}.
%	\item \underline{Phuong T. Nguyen}, Loan T. Phan ``\emph{A Proposed Architecture for the Realization and Management of an Information-Centric Network,}'' in Proceedings of the first NAFOSTED Conference on Information and Computer Science, NICS 2014.
%	\item \underline{Phuong T. Nguyen}, Volkmar Schau, Wilhelm Rossak, ``\emph{Mobile Agent Communication in Highly Dynamic Networks: A Self-Adaptive Architecture inspired by the Honey Bee Colony},'' EAI Endorsed Transactions on Context-aware Systems and Applications, DOI: \href{http://dx.doi.org/10.4108/casa.1.2.e5}{10.4108/casa.1.2.e5}.
%	\item \underline{Phuong T. Nguyen}, ``\emph{Building Consensus in Context-Aware Systems Using Ben-Or’s Algorithm: Some Proposals for Improving the Convergence Speed},'' in Proceedings of the International Conference on Context-Aware Systems and Applications, ICCASA 2013, DOI: \href{https://link.springer.com/chapter/10.1007/978-3-319-05939-6\_9}{10.1007/978-3-319-05939-6\_9}.
%	\item \underline{Phuong T. Nguyen}, Volkmar Schau, Wilhelm Rossak ``\emph{A Context-Aware Model for the Management of Agent Platforms in Dynamic Networks},'' in Proceedings of the International Conference on Context-Aware Systems and Applications, ICCASA 2013, DOI: \href{https://doi.org/10.1007/978-3-319-05939-6\_8}{10.1007/978-3-319-05939-6\_8}.	
%	\item \underline{Phuong T. Nguyen}, Volkmar Schau, Wilhelm Rossak, ``\emph{Performance comparison of some message transport protocol implementations for agent community communication},'' in Proceedings of the 11th International Conference on Innovative Internet Community Systems, I2CS 2011, ISBN 978-3-88579-280-2.		
%	\item \underline{Phuong T. Nguyen}, Volkmar Schau, Wilhelm Rossak ``\emph{An Adaptive Communication Model for Mobile Agents inspired by the Honey Bee Colony: Theory and Evaluation},'' in Proceedings of the 10th European Workshop on Multiagent Systems, EUMAS 2012.
%	\item \underline{Phuong T. Nguyen}, Volkmar Schau, Wilhelm Rossak ``\emph{Towards an Adaptive Communication Model for Mobile Agents in Highly Dynamic Networks based on Swarming Behaviour},'' in Proceedings of the 9th European Workshop on Multi-agent Systems, EUMAS 2011.
%\end{itemize}
%\vspace{-.2cm}
%\smallskip

%\subsection*{Posters}
%
%\begin{itemize}
%\item Claudio Di Sipio, Davide Di Ruscio, \underline{Phuong T. Nguyen}, ``\emph{Using the Low-code Paradigm to Support the Development of Recommender Systems},'' Poster session at the 23rd ACM/\-IEEE International Conference on Model Driven Engineering Languages and Systems (MODELS 2020).
%\end{itemize}

%``,''

%\subsection*{Manuscripts under review/revision}
%\vspace{-.2cm}

%\begin{itemize}
%	\item \underline{Phuong T. Nguyen}, Juri Di Rocco, Ludovico Iovino, Davide Di Ruscio, Alfonso Pierantonio, ``\emph{Evaluation of Machine Learning Classifiers for Metamodels},'' submitted to Springer Software and Systems Modeling (SoSyM), under review after revision.
%	\item \underline{Phuong T. Nguyen}, Juri Di Rocco, Riccardo Rubei, Claudio Di Sipio, Davide Di Ruscio, ``\emph{DeepLib: Recommending Third-party Library Updates with Deep Neural Networks},'' submitted to Elsevier Journal of Systems and Software (JSS), under review.
%	\item Juri Di Rocco, Davide Di Ruscio, Claudio Di Sipio, \underline{Phuong T. Nguyen}, Riccardo Rubei, ``\emph{A Hybrid Recommender System for Labeling GitHub and Maven
%		Repositories},'' submitted to Springer Empirical Software Engineering (EMSE), under review.
%	\item Juri Di Rocco, Davide Di Ruscio, Claudio Di Sipio, \underline{Phuong T. Nguyen}, Alfonso Pierantonio, ``\emph{MemoRec: A Recommender System for Assisting Modelers in
%		Specifying Metamodels},'' submitted to Springer Software and Systems Modeling (SoSyM), under review.
%	\item Linh T. Duong, \underline{Phuong T. Nguyen}, Ludovico Iovino, and Michele Flammini, ``\emph{Automatic Detection of Covid-19 from Chest X-ray and Lung Computed Tomography Images using Deep Neural Networks and Transfer Learning},'' submitted to Elsevier Applied Soft Computing (ASOC), under review.
%	%\item \underline{Linh T. Duong}, Phuong T. Nguyen, Ludovico Iovino, and Michele Flammini, Deep Learning for Automated Recognition of Covid-19 from Chest X-ray Images, \emph{MedRxiv preprint} \url{doi: https://doi.org/10.1101/2020.08.13.20173997}
%	%\item Linh T. Duong, Nhi H. Le, Toan B. Tran, Phuong T. Nguyen, and Vuong M. Ngo, Detection of Tuberculosis from Chest X-ray Images: Boosting the Performance with Vision Transformer and Transfer Learning, a manusc\-ript to be submitted to \emph{Elsevier Applied Soft Computing}
%	\item Linh T. Duong, Nhi H. Le, Toan B. Tran, Vuong M. Ngo, and \underline{Phuong T. Nguyen}, ``\emph{Detection of Tuberculosis from Chest X-ray Images: Boosting the Performance with Vision Transformer and Transfer Learning},'' submitted to Elsevier Expert Systems with Applications (ESWA), under review.
%	%\item Linh T. Duong, Nhi H. Le, Toan B. Tran, Vuong M. Ngo, and Phuong T. Nguyen, ``Deep Learning for Automated Recognition of Bacterial and Viral Pneumonia from Chest X-ray Images,'' a manuscript to be submitted to \emph{Elsevier Applied Soft Computing}
%	\item Linh T. Duong, Nhi H. Le, Toan B. Tran, Vuong M. Ngo, and \underline{Phuong T. Nguyen}, ``\emph{Applying EfficientNet and Transfer Learning for Automatic Weed Detection},'' submitted to Elsevier Computers and Electronics in Agriculture (COMPAG), under revision.
%%	\item Linh T. Duong, Nhi H. Le, Toan B. Tran, Vuong M. Ngo, and \underline{Phuong T. Nguyen}, ``\emph{Applying EfficientNet and Transfer Learning for Automatic Weed Detection},'' submitted to Elsevier Computers and Electronics in Agriculture (COMPAG), under review.
%\end{itemize}
%\end{}

%\subsection*{Pre-prints}
%\begin{itemize}
%	\item Linh T. Duong, \underline{Phuong T. Nguyen}, Ludovico Iovino, and Michele Flammini, ``\emph{Deep Learning for Automatic Detection of Covid-19 from Chest X-ray Images},'' MedrXiv.
%\end{itemize}

%\printbibliography

%\bibliographystyle{apalike}
%\bibliographystyle{abbrv}
%\bibliographystyle{elsarticle-num}
%\bibliographystyle{ACM-Reference-Format}

%\bibliographystyle{unsrt}
%\bibliography{main}

%\clearpage	

%which authors who contributed the most are listed first


%\emph{%Unless otherwise stated, 
%%	The publications follow a contributed-based norm, \ie authors who contributed the most are listed first. %\ie the first author contributes the most, unless otherwise stated; 
%	Corresponding authors in journal articles are marked with an asterisk ().	}
%\vspace{-.2cm}

\begin{thebibliography}{100}
	
%
\section{10 Most Important Publications}%\footnote{Papers are contributed-based}
\vspace{.2cm}

%\begin{thebibliography}{100}
%	\bibitem{one} Angela Barriga, Davide Di Ruscio, Ludovico Iovino, \underline{Phuong T. Nguyen}, Alfonso  Pierantonio ``\emph{An extensible tool-chain for analyzing datasets of metamodels},'' in Proceedings of the 23rd ACM/IEEE International Conference on Model Driven Engineering Languages and Systems: Companion Proceedings, DOI: \href{https://doi.org/10.1145/3417990.3419626}{10.1145/3417990.3419626}.	

%\newrefcontext[labelprefix=J]


%====================
%\begin{enumerate}[label=(\alph*)]
%	\item The first item
%	\begin{enumerate}[label*=\arabic*.]
%		\item Nested item 1
%		\item Nested item 2
%	\end{enumerate}
%	\item The second item
%	\item The third etc \ldots
%\end{enumerate}
%====================



%====================
%\label{The Selected Papers}
%{\small	
%	\begin{enumerate}[label*=\arabic*.] %[labelindent=-5pt,label={[P}{\arabic*]}]
%		\item \label{P1}
%		F. Zampetti, C. Noiseux, G. Antoniol, F. Khomh, and M. Di Penta. Recommending when design technical debt should be self-admitted. In 2017 IEEE International Conference on Software Maintenance and Evolution (ICSME), pages 216–226, 2017. doi: 10.1109/ICSME.2017.44.
%	\end{enumerate}
%}
%====================

%\subsection*{Book chapters}
%
%	\bibitem{NGUYEN2023BookChapter} Davide Di Ruscio, \underline{Phuong T. Nguyen}, Alfonso Pierantonio, ``\emph{Machine Learning for Managing Modeling Ecosystems: Techniques, Applications, and a Research Vision},'' Software Ecosystems: Tooling and Analytics, Springer 2023, DOI: \href{http://dx.doi.org/10.1007/978-3-031-36060-2\_10}{http://dx.doi.org/10.1007/978-3-031-36060-2\_10}.	
%

%\subsection*{Journal papers}

%	\bibitem{Bucaioni2024} 
%	Alessio Bucaioni, Hampus Ekedahl, Vilma Helander, \underline{Phuong T. Nguyen}, ``\emph{Programming with ChatGPT: How far can we go?,}'' Elsevier Machine Learning with Applications (MLWA), 2024, ISSN: 2666-8270, DOI: \href{https://doi.org/10.1016/j.mlwa.2024.100526}{https://doi.org/10.1016/j.mlwa.2024.100526}.
	
	
%	\bibitem{DiRuscio2022} Davide Di Ruscio, Paola Inverardi, Patrizio Migliarini, \underline{Phuong T. Nguyen}, ``\emph{Leveraging Privacy Profiles to Empower Users in the Digital Society},'' Springer Automated Software Engineering, to appear.

	\bibitem{NguyenESWALUPE}  %\textbf{Journal manuscript}:\\  %\textbf{Journal paper}:\\	
	\underline{Phuong T. Nguyen}, Claudio Di Sipio, Juri Di Rocco, Riccardo Rubei, Davide Di Ruscio, Massimiliano Di Penta ``\emph{Fitting Missing API Puzzles with Machine Translation Techniques},'' Elsevier Expert Systems with Applications (ESWA), 2023, ISSN: 0957-4174, DOI: \href{https://doi.org/10.1016/j.eswa.2022.119477}{10.1016/j.eswa.2022.119477}. 
	

	\bibitem{NGUYEN2022117267} %\textbf{Journal manuscript}:\\
	\underline{Phuong T. Nguyen}, Juri Di Rocco, Riccardo Rubei, Claudio Di Sipio, Davide Di Ruscio, ``\emph{DeepLib: Machine Translation Techniques to Recommend Upgrades for Third-party Libraries},'' Elsevier Expert Systems with Applications (ESWA), 2022, ISSN: 0957-4174, DOI: \href{https://doi.org/10.1016/j.eswa.2022.117267}{10.1016/j.eswa.2022.117267}.
	%	\emph{accepted for publication}.

%	\bibitem{NguyenSoSyM2021} %\textbf{Journal paper}:\\ 
%	\underline{Phuong T. Nguyen}, Juri Di Rocco, Ludovico Iovino, Davide Di Ruscio, Alfonso Pierantonio, ``\emph{Evaluation of Machine Learning Classifiers for Metamodels},'' invited paper to Springer Software and Systems Modeling (SoSyM), DOI: \href{https://doi.org/10.1007/s10270-021-00913-x}{https://doi.org/10.1007/s10270-021-00913-x}.%\emph{to appear}.
%%	
%	\bibitem{DiSipio2023} %\textbf{Journal manuscript}:\\
%	Claudio Di Sipio, Juri Di Rocco, Davide Di Ruscio, \underline{Phuong T. Nguyen}, ``\emph{MORGAN: a modeling recommender system based on graph kernel},'' an invited paper to the Special Issue for MODELS 2022, Springer Software and Systems Modeling (SoSyM), DOI: \href{https://doi.org/10.1007/s10270-023-01102-8}{https://doi.org/10.1007/s10270-023-01102-8}.
%	
%	\bibitem{DiSipio2023} %\textbf{Journal manuscript}:\\
%	Claudio Di Sipio, Juri Di Rocco, Davide Di Ruscio, \underline{Phuong T. Nguyen}, ``\emph{LEV4REC:  A feature-based approach to engineering RSSEs},'' Elsevier Journal of Computer Languages (COLA), DOI: \href{https://doi.org/10.1016/j.cola.2023.101256}{https://doi.org/10.1016/j.cola.2023.101256}.
	
%	\bibitem{DuongSOCO2023} Linh T. Duong, Nhi H. Le, Toan B. Tran, Vuong M. Ngo, and \underline{Phuong T. Nguyen}, ``\emph{Automatic Detection of Weeds: Synergy between EfficientNet and Transfer Learning to Enhance the Prediction Accuracy},'' Springer Soft Computing (SOCO), DOI: \href{https://doi.org/10.1007/s00500-023-09212-7}{https://doi.org/10.1007/s00500-023-09212-7}.%, \emph{to appear}.
		
	
%	\bibitem{DUONG2023120107} %\textbf{Journal manuscript}:\\
%	Linh T. Duong, Thu T. H. Doan, Cong Q. Chu, \underline{Phuong T. Nguyen}, ``\emph{Fusion of edge detection and graph neural networks to classifying electrocardiogram signals},'' Elsevier Expert Systems with Applications (ESWA), 2023, ISSN: 0957-4174, DOI: \href{https://doi.org/10.1016/j.eswa.2023.120107}{https://doi.org/10.1016/j.eswa.2023.120107}.
%	%	\emph{accepted for publication}.
	

	
%	\bibitem{DuongASOC2023} Linh T. Duong, Cong Q. Chu, \underline{Phuong T. Nguyen}, Son T. Nguyen, and Binh Q. Tran, ``\emph{Edge detection and graph neural networks to classify mammograms: A case study with a dataset from Vietnamese patients},'' Elsevier Applied Soft Computing (ASOC), DOI: \href{https://doi.org/10.1016/j.asoc.2022.109974}{https://doi.org/10.1016/j.asoc.2022.109974}.
	
	
	
	
%	\bibitem{DuongASOC2022} Linh T. Duong, \underline{Phuong T. Nguyen}, Ludovico Iovino, and Michele Flammini, ``\emph{Automatic Detection of Covid-19 from Chest X-ray and Lung Computed Tomography Images using Deep Neural Networks and Transfer Learning},'' Elsevier Applied Soft Computing (ASOC), DOI: \href{https://doi.org/10.1016/j.asoc.2022.109851}{https://doi.org/10.1016/j.asoc.2022.109851}.
		
	\bibitem{9359479}  %\textbf{Journal paper}:\\	
	\underline{Phuong T. Nguyen}, Juri Di Rocco, Claudio Di Sipio, Davide Di Ruscio, Massimiliano Di Penta ``\emph{Recommending API Function Calls and Code Snippets to Support Software Development},'' IEEE Transactions on Software Engineering (TSE), 2021, ISSN: 1939-3520, DOI: \href{https://doi.org/10.1109/TSE.2021.3059907}{10.1109/TSE.2021.3059907}.%, preprint: \href{https://arxiv.org/abs/2102.07508}{https://arxiv.org/abs/2102.07508}.	
	
%	\bibitem{DiRoccoAPIN2022} Juri Di Rocco, Davide Di Ruscio, Claudio Di Sipio, \underline{Phuong T. Nguyen}, Riccardo Rubei, ``\emph{HybridRec: A Recommender System for Tagging GitHub Repositories},'' Springer Applied Intelligence (APIN), ISSN: 1573-7497, DOI: \href{https://doi.org/10.1007/s10489-022-03864-y}{https://doi.org/10.1007/s10489-022-03864-y}.
	
%	\bibitem{DiRoccoSoSyM2022} Juri Di Rocco, Davide Di Ruscio, Claudio Di Sipio, \underline{Phuong T. Nguyen}, Alfonso Pierantonio, ``\emph{MemoRec: A Recommender System for Assisting Modelers in Specifying Metamodels},'' Springer Software and Systems Modeling (SoSyM), DOI: \href{https://doi.org/10.1007/s10270-022-00994-2}{https://doi.org/10.1007/s10270-022-00994-2}.
	
%	\bibitem{DuongESWA2021} %\textbf{Journal paper}:\\ 
%	Linh T. Duong, Nhi H. Le, Toan B. Tran, Vuong M. Ngo, \underline{Phuong T. Nguyen}, ``\emph{Detection of Tuberculosis from Chest X-ray Images: Boosting the Performance with Vision Transformer and Transfer Learning},'' Elsevier Expert Systems with Applications (ESWA), Volume 184, 2021, ISSN: 0957-4174, DOI: 	\href{https://doi.org/10.1016/j.eswa.2021.115519}{https://doi.org/10.1016/j.eswa.2021.115519}.
	
%	\bibitem{RubeiEASE2021} %\textbf{Journal manuscript}:\\ 
%	Riccardo Rubei, Claudio Di Sipio, Davide Di Ruscio, \underline{Phuong T. Nguyen}, Juri Di Rocco ``\emph{Providing Upgrade Plans for Third-party Libraries: A Recommender System using Migration Graphs},'' Springer Applied Intelligence (APIN), ISSN: 1573-7497, DOI: \href{https://doi.org/10.1007/s10489-021-02911-4}{https://doi.org/10.1007/s10489-021-02911-4}.
	
%	
%	\bibitem{JOCCH2021}  %\textbf{Journal paper}:\\	
%	Vuong M. Ngo, Van T. Duong, Tat-Bao-Thien Nguyen, \underline{Phuong T. Nguyen}, Owen Colan ``\emph{An Efficient Classification Algorithm for Traditional Textile Patterns from Different Cultures based on Hypergraph Structures},'' ACM Journal on Computing and Cultural Heritage (JOCCH), 2021, DOI: \href{https://doi.org/10.1145/3465381}{https://doi.org/10.1145/3465381}.
%	
	
%	\bibitem{DiRoccoEMSE2020} %\textbf{Journal paper}:\\	
%	Juri Di Rocco, Davide Di Ruscio, Claudio Di Sipio, \underline{Phuong T. Nguyen}, Riccardo Rubei, ``\emph{Development of recommendation systems for software engineering: the	CROSSMINER experience},'' Springer Empirical Software Engineering (EMSE), ISSN: 1573-7616, DOI: \href{https://doi.org/10.1007/s10664-021-09963-7}{https://doi.org/10.1007/s10664-021-09963-7}, preprint: \href{https://arxiv.org/abs/2103.06987}{https://arxiv.org/abs/2103.06987}. %(\emph{The authors are listed in alphabetical order}).		
%	
%	\bibitem{9345512} %\textbf{Journal paper}:\\
%	Ludovio Iovino, \underline{Phuong T. Nguyen}, Amleto Di Salle, Francesco Gallo, Michele Flammini ``\emph{Unavailable Transit Feed Specification: Making It Available With Recurrent Neural Networks},'' IEEE Transactions on Intelligent Transportation Systems (T-ITS), 2021, ISSN: 1558-0016, DOI: \href{https://doi.org/10.1109/TITS.2021.3053373}{https://doi.org/10.1109/TITS.2021.3053373}.	
%	
	\bibitem{NGUYEN2021110860} %\textbf{Journal paper}:\\
	\underline{Phuong T. Nguyen}, Davide Di Ruscio, Alfonso Pierantonio, Juri Di Rocco, Ludovico Iovino, ``\emph{Convolutional neural networks for enhanced classification mechanisms of metamodels},'' Elsevier Journal of Systems and Software (JSS), 2020, ISSN: 0164-1212, DOI: \href{https://doi.org/10.1016/j.jss.2020.110860}{10.1016/j.jss.2020.110860}.		
	%	\item Phuong T. Nguyen, Juri Di Rocco, Davide Di Ruscio, Massimiliano Di Penta, ``\emph{CrossRec: Supporting Software Developers by Recommending Third-party Libraries},'' Journal of Systems and Software,	
%	\bibitem{CAPILUPPI2020106279} %\textbf{Journal paper}:\\
%	Andrea Capiluppi, Davide Di Ruscio, Juri Di Rocco, \underline{Phuong T. Nguyen}, Nemitari Ajienka, ``\emph{Detecting Java Software Similarities by using Different Clustering Techniques},'' Elsevier Information and Software Technology (IST), 2020, ISSN: 0950-5849, DOI: \href{https://doi.org/10.1016/j.infsof.2020.106279}{https://doi.org/10.1016/j.infsof.2020.106279}.		
	
	
%	\bibitem{RUBEI2020106367} %%\textbf{Journal paper}:\\
%	Riccardo Rubei, Claudio Di Sipio, \underline{Phuong T. Nguyen}, Juri Di Rocco, Davide Di Ruscio, ``\emph{PostFinder: Mining Stack Overflow posts to support software developers},'' Elsevier Information and Software and Technology (IST), 2020, ISSN: 0950-5849, DOI: \href{https://doi.org/10.1016/j.infsof.2020.106367}{https://doi.org/10.1016/j.infsof.2020.106367}.


	
%	\bibitem{DUONG2020105326} %%\textbf{Journal paper}:\\
%	Linh T. Duong, \underline{Phuong T. Nguyen}, Claudio Di Sipio, and Davide Di Ruscio, ``\emph{Automated Fruits Recognition on using EfficientNet and MixNet},'' Elsevier Computers and Electronics in Agriculture, Volume 171, April 2020, 105326, ISSN: 0168-1699, DOI: \href{https://doi.org/10.1016/j.compag.2020.105326}{https://doi.org/10.1016/j.compag.2020.105326}.	
%	
	\bibitem{NGUYEN2020110460} %\textbf{Journal paper}:\\
	\underline{Phuong T. Nguyen}, Juri Di Rocco, Davide Di Ruscio, Massimiliano Di Penta, ``\emph{CrossRec: Supporting Software Developers by Recommending Third-party Libraries},'' Elsevier Journal of Systems and Software (JSS), 2020, ISSN: 0164-1212, DOI: \href{https://doi.org/10.1016/j.jss.2019.110460}{10.1016/j.jss.2019.110460}.	
	
	\bibitem{DBLP:journals/sqj/NguyenRRR20} %\textbf{Journal paper}:\\
	\underline{Phuong T. Nguyen}, Juri Di Rocco, Riccardo Rubei, Davide Di Ruscio, ``\emph{An Automated Approach to Assess the Similarity of GitHub Repositories},'' Springer Software Quality Journal (SQJ), Vol 28, pages 595–631, 2020, ISSN: 0963-9314, DOI: \href{https://doi.org/10.1007/s11219-019-09483-0}{10.1007/s11219-019-09483-0}.


%	\bibitem{10.4108/casa.1.2.e5} %\textbf{Journal paper}: 
%	\underline{Phuong T. Nguyen}, Volkmar Schau, Wilhelm Rossak, ``\emph{Mobile Agent Communication in Highly Dynamic Networks: A Self-Adaptive Architecture inspired by the Honey Bee Colony},'' EAI Endorsed Transactions on Context-aware Systems and Applications, ISSN: 2409-0026, DOI: \href{http://dx.doi.org/10.4108/casa.1.2.e5}{http://dx.doi.org/10.4108/casa.1.2.e5}.



%	\subsection*{Conference and Workshop papers}
	
	\bibitem{NguyenMSR2023} %\textbf{Paper in conference proceedings}:\\ 
	\underline{Phuong T. Nguyen}, Riccardo Rubei, Juri Di Rocco, Claudio Di Sipio, Davide Di Ruscio, Massimiliano Di Penta, ``\emph{Dealing with Popularity Bias in Recommender Systems for Third-party Libraries: How far Are We?},'' in Proceedings %accepted as a full research paper at 
	of the 20th International Conference on Mining Software Repositories, MSR 2023, %\emph{to appear}. %to appear. %, 
	DOI: \href{https://doi.org/10.1109/MSR59073.2023.00016}{10.1109/MSR59073.2023.00016}.%\emph{to appear}.
	
%	 \bibitem{DoanEASE2023} Thu T. H. Doan, \underline{Phuong T. Nguyen}, Juri Di Rocco, Davide Di Ruscio. ``\emph{Too long; didn’t read: Automatic summarization of GitHub
%		README.MD with Transformers},'' in Proceedings of the International Conference on Evaluation and Assessment in Software Engineering 2023 (EASE 2023), DOI: \href{https://doi.org/10.1145/3593434.3593448}{https://doi.org/10.1145/3593434.3593448}.
	
%	\bibitem{HoEASE2023} Anh Ho, Anh M. T. Bui, \underline{Phuong T. Nguyen}, Amleto Di Salle. ``\emph{Fusion of deep convolutional and LSTM recurrent neural networks for automated detection of code smells},'' in Proceedings of the International Conference on Evaluation and Assessment in Software Engineering 2023 (EASE 2023), DOI: \href{https://doi.org/10.1145/3593434.359347}{https://doi.org/10.1145/3593434.3593478}.
	
%	\bibitem{Schulz2023} Jonas Schulz, Hristina Radak, \underline{Phuong T. Nguyen}, Giang T. Nguyen, Frank H. P. Fitzek. ``\emph{On the Limits of Lossy Compression for Human Activity Recognition in Sensor Networks},'' in Proceedings of the IEEE 48th Conference on Local Computer Networks (LCN 2023), DOI: \href{https://doi.org/10.1109/LCN58197.2023.10223374}{https://doi.org/10.1109/LCN58197.2023.10223374}.
%	
		
%	\bibitem{DiRoccoMODELS2022} %\textbf{Paper in conference proceedings}:\\ 
%	Juri Di Rocco, Claudio Di Sipio, \underline{Phuong T. Nguyen}, Davide Di Ruscio, Alfonso Pierantonio ``\emph{Finding with NEMO: A Recommender System to Forecast the Next Modeling Operations},'' in Proceedings of the 25th International Conference on Model Driven Engineering Languages and Systems, MODELS 2022, DOI: \href{https://doi.org/10.1145/3550355.3552459}{https://doi.org/10.1145/3550355.3552459}.
%	
	
%	\bibitem{DiSalleTechDebt2022} Amleto Di Salle, Alessandra Rota, \underline{Phuong T. Nguyen}, Davide Di Ruscio, Francesca Arcelli Fontana, Irene Sala, ``\emph{PILOT: Synergy between Text Processing and Neural Networks to Detect Self-Admitted Technical Debt},'' in Proceedings of the 5th International Conference on Technical Debt, TechDebt 2022 co-located with ICSE 2022, DOI: \href{https://doi.org/10.1145/3524843.3528093}{https://doi.org/10.1145/3524843.3528093}.%\emph{to appear}.
%	 
%	\bibitem{RubeiSANER2022} Riccardo Rubei, Davide Di Ruscio, Claudio Di Sipio, Juri Di Rocco, \underline{Phuong T. Nguyen}, ``\emph{Endowing third-party libraries recommender systems with explicit user feedback mechanism},'' in Proceedings of the 29th IEEE International Conference on Software Analysis, Evolution and Reengineering, SANER 2022, DOI: \href{https://doi.org/10.1109/SANER53432.2022.00099}{https://doi.org/10.1109/SANER53432.2022.00099}
	
	
	
	
	%.\emph{to appear}.
	
	\bibitem{NguyenASE2021} %\textbf{Paper in conference proceedings}:\\ 
	\underline{Phuong T. Nguyen}, Juri Di Rocco, Claudio Di Sipio, Davide Di Ruscio, Massimiliano Di Penta, ``\emph{Adversarial Attacks to API Recommender Systems: Time to Wake Up and Smell the Coffee?},'' in Proceedings %accepted as a full research paper at 
	of the 36th IEEE/ACM International Conference on Automated Software Engineering, ASE 2021, DOI: \href{https://doi.org/10.1109/ASE51524.2021.9678946}{10.1109/ASE51524.2021.9678946}.%\emph{to appear}.
		
	\bibitem{NguyenEASE2021} %\textbf{Paper in conference proceedings}:\\ 
	\underline{Phuong T. Nguyen}, Juri Di Rocco, Claudio Di Sipio, Davide Di Ruscio, Massimiliano Di Penta, ``\emph{Adversarial Machine Learning: On the Resilience of Third-party Library Recommender Systems},'' in Proceedings of the 25th International Conference on Evaluation and Assessment in Software Engineering, EASE 2021, DOI: \href{https://doi.org/10.1145/3463274.3463809}{10.1145/3463274.3463809}.
	
%	\bibitem{NguyenIIR2021}%\textbf{Paper in conference proceedings}:\\
%	\underline{Phuong T. Nguyen}, Juri Di Rocco, Riccardo Rubei, Claudio Di Sipio, Davide Di Ruscio, ``\emph{Recommending Third-party Library Updates with LSTM Neural Networks},'' in Proceedings of the 11th Italian Information Retrieval Workshop, September 13--15, 2021, Bari, Italy.
%	
%	\bibitem{DiSipioMODELS2021} %\textbf{Paper in conference proceedings}:\\ 
%	Juri Di Rocco, Claudio Di Sipio, Davide Di Ruscio, \underline{Phuong T. Nguyen}, ``\emph{A GNN-based Recommender System to Assist the Specification of Metamodels and Models},'' in Proceedings of the 24th International Conference on Model Driven Engineering Languages and Systems, MODELS 2021, DOI: \href{https://doi.org/10.1109/MODELS50736.2021.00016}{https://doi.org/10.1109/MODELS50736.2021.00016}.
%		
%	\bibitem{KaRS2021}%\textbf{Paper in conference proceedings}:\\
%	Juri Di Rocco, Claudio Di Sipio, Davide Di Ruscio, \underline{Phuong T. Nguyen}, Claudio Pomo, ``\emph{On the Need for a Body of Knowledge on Recommender Systems},'' in Proceedings of the 3rd Edition of Knowledge-aware and Conversational Recommender Systems (KaRS) \& 5th Edition of Recommendation in Complex Environments (ComplexRec) Joint Workshop @ RecSys 2021, September 27--October 1, 2021, Amsterdam, Netherlands.
		
%	\bibitem{OpenMBEE2021} %\textbf{Paper in conference proceedings}:\\
%	Riccardo Rubei, Juri Di Rocco, Davide Di Ruscio, \underline{Phuong T. Nguyen}, Alfonso Pierantonio, ``\emph{A Lightweight Approach for the Automated Classification and Clustering of Metamodels},'' in Proceedings of the 2nd International Workshop on Open Model Based Engineering Environment (OpenMBEE 2021), co-located with the IEEE/ACM 24th International Conference on Model Driven Engineering Languages and Systems, MODELS 2021, DOI: \href{https://doi.org/10.1109/MODELS-C53483.2021.00074}{https://doi.org/10.1109/MODELS-C53483.2021.00074}. 
	
%	\bibitem{10.1145/3417990.3420202} %\textbf{Paper in conference proceedings}:\\
%	Claudio Di Sipio, Davide Di Ruscio, \underline{Phuong T. Nguyen} ``\emph{Democratizing the development of recommender systems by means of low-code platforms},'' in Proceedings of the 23rd ACM/IEEE International Conference on Model Driven Engineering Languages and Systems: Companion Proceedings, MODELS 2020, DOI: \href{https://doi.org/10.1145/3417990.3420202}{https://doi.org/10.1145/3417990.3420202}.
%		
%	\bibitem{10.1145/3383219.3383227}%\textbf{Paper in conference proceedings}:\\
%	Claudio Di Sipio, Riccardo Rubei, Davide Di Ruscio, \underline{Phuong T. Nguyen}, ``\emph{A Multinomial Naïve Bayesian (MNB) Network to Automatically Recommend Topics for GitHub Repositories},'' in Proceedings of the 24th International Conference on Evaluation and Assessment in Software Engineering, EASE 2020, ISBN:	978-1-4503-7731-7, DOI: \href{https://doi.org/10.1145/3383219.3383227}{https://doi.org/10.1145/3383219.3383227}.	
%	
%	\bibitem{10.1145/3417990.3419626} %\textbf{Paper in conference proceedings}:\\
%	Angela Barriga,  Davide Di Ruscio, Ludovico Iovino, \underline{Phuong T. Nguyen}, Alfonso Pierantonio, ``\emph{An Extensible Tool-Chain for Analyzing Datasets of Metamodels},'' in Proceedings of the 23rd ACM/IEEE International Conference on Model Driven Engineering Languages and Systems: Companion Proceedings, MODELS 2020, %Association for Computing Machinery, 
%	DOI: \href{https://doi.org/10.1145/3417990.3419626}{https://doi.org/10.1145/3417990.3419626}.%  (\emph{The authors are listed in alphabetical order}). 	
			
	\bibitem{8812051} %\textbf{Paper in conference proceedings}:\\
	\underline{Phuong T. Nguyen}, Juri Di Rocco, Davide Di Ruscio, Lina Ochoa, Thomas Degueule, Massimiliano Di Penta, ``\emph{FOCUS: A Recommender System for Mining API Function Calls and Usage Patterns},'' in Proceedings of the 41st International Conference on Software Engineering, ICSE 2019, ISBN: 978-1-7281-0869-8, DOI: \href{https://doi.org/10.1109/ICSE.2019.00109}{10.1109/ICSE.2019.00109}. %Lecture Notes in Computer Science, Springer. \href{\url{https://doi.org/10.1109/ICSE.2019.00109}}		
	
%	\bibitem{DBLP:conf/caise/NguyenRR19}%\textbf{Paper in conference proceedings}:\\
%	\underline{Phuong T. Nguyen}, Juri Di Rocco, Davide Di Ruscio ``\emph{Building Information Systems Using Collaborative-Filtering Recommendation Techniques},'' in Proceedings of Advanced Information Systems Engineering Workshops, CAiSE 2019, ISBN: 978-3-030-21290-2, DOI: \href{https://doi.org/10.1007/978-3-030-20948-3\_19}{https://doi.org/10.1007/978-3-030-20948-3\_19}.
		
%	\bibitem{8906979}%\textbf{Paper in conference proceedings}:\\
%	\underline{Phuong T. Nguyen}, Juri Di Rocco, Davide Di Ruscio, Alfonso Pierantonio, Ludovico Iovino, ``\emph{Automated Classification of Metamodel Repositories: A Machine Learning Approach},'' in Proceedings of the 22nd ACM/IEEE International Conference on Model Driven Engineering Languages and Systems, MODELS 2019, DOI: \href{https://doi.org/10.1109/MODELS.2019.00011}{https://doi.org/10.1109/MODELS.2019.00011}. %Lecture Notes in Computer Science, Springer. 	
	
	
%	
%	\bibitem{10.1145/3319008.3319353}%\textbf{Paper in conference proceedings}:\\
%	\underline{Phuong T. Nguyen}, Juri Di Rocco, Davide Di Ruscio: ``\emph{Enabling heterogeneous recommendations in OSS development: What's done and what's next in CROSSMINER},'' In Proceedings of the 23rd International Conference on Evaluation and Assessment on Software Engineering, EASE 2019, ISBN:	978-1-4503-7145-2, DOI: \href{https://doi.org/10.1145/3319008.3319353}{https://doi.org/10.1145/3319008.3319353}. %Lecture Notes in Computer Science, Springer. \href{\url{https://doi.org/10.1109/ICSE.2019.00109}}	
%	

%	\bibitem{8498236}%\textbf{Paper in conference proceedings}:\\
%	\underline{Phuong T. Nguyen}, Juri Di Rocco, Riccardo Rubei, Davide Di Ruscio, ``\emph{CrossSim: exploiting mutual relationships to detect similar OSS projects},'' in Proceedings of the 44th Euromicro Conference on	Software Engineering and Advanced Applications, SEAA 2018, ISBN: 978-1-5386-7383-6, DOI: \href{https://doi.org/10.1109/SEAA.2018.00069}{https://doi.org/10.1109/SEAA.2018.00069}.		
	
%	\bibitem{10.1145/3382494.3410690} %\textbf{Paper in conference proceedings}:\\
%	Juri Di Rocco, Davide Di Ruscio, Claudio Di Sipio, \underline{Phuong Nguyen} and Riccardo Rubei, ``\emph{TopFilter: An Approach to Recommend Relevant GitHub Topics},'' in Proceedings of the 14th International Symposium on Empirical Software Engineering and Measurement, ESEM 2020, ISBN:	
%	978-1-4503-7580-1, DOI: \href{https://doi.org/10.1145/3382494.3410690}{https://doi.org/10.1145/3382494.3410690}.	
	
%		\bibitem{NguyenKars2018} %\textbf{Paper in conference proceedings}:\\ 
%	\underline{Phuong T. Nguyen}, Juri Di Rocco, Davide Di Ruscio ``\emph{Knowledge-aware Recommender System for Software Development},'' in Proceedings of the Knowledge-aware and Conversational Recommender Systems Workshop, KaRS 2018 (co-located with RecSys 2018), October 7, 2018, Vancouver, Canada \href{http://ceur-ws.org/Vol-2290/kars2018_paper4.pdf}{(http://ceur-ws.org/Vol-2290/kars2018\_paper4.pdf)}.	
	
%	\bibitem{NguyenIIR2018} %\textbf{Paper in conference proceedings}:\\
%	\underline{Phuong T. Nguyen}, Juri Di Rocco, Davide Di Ruscio ``\emph{Mining Software Repositories to Support OSS Developers: A Recommender Systems Approach},'' in Proceedings of the Italian Information Retrieval Workshop, IIR 2018 \href{http://ceur-ws.org/Vol-2140/paper9.pdf}{(http://ceur-ws.org/Vol-2140/paper9.pdf)}.
	
%	\bibitem{DBLP:conf/ismis/NguyenERN17} %\textbf{Paper in conference proceedings}:\\ 
%	\underline{Phuong T. Nguyen}, Kai Eckert, Azzurra Ragone, Tommaso Di Noia, ``\emph{Modification to K-Medoids and CLARA for Effective Document Clustering},'' in Proceedings of the 23rd International Symposium on Methodologies for Intelligent Systems, ISMIS 2017, DOI: \href{https://doi.org/10.1007/978-3-319-60438-1\_47}{https://doi.org/10.1007/978-3-319-60438-1\_47}. 	
		
%	\bibitem{DBLP:conf/fdse/NguyenL15} %\textbf{Paper in conference proceedings}:\\ 
%	\underline{Phuong T. Nguyen}, Hong Anh Le ``\emph{Finding Similar Artists from the Web of Data: A PageRank Based Semantic Similarity Metric},'' in Proceedings of the 2nd International Conference on Future Data and Security Engineering, FDSE 2015, ISBN: 978-3-319-26134-8, DOI: \href{https://doi.org/10.1007/978-3-319-26135-5\_8}{https://doi.org/10.1007/978-3-319-26135-5\_8}.	

%	\bibitem{10.1145/2740908.2742141} %\textbf{Paper in conference proceedings}:\\ 
%	\underline{Phuong T. Nguyen}, Paolo Tomeo, Tommaso Di Noia, Eugenio Di Sciascio: ``\emph{An evaluation of SimRank and Personalized PageRank to build a recommender system for the Web of Data},'' in Proceedings of the 7th International Workshop on Web Intelligence \& Communities, at WWW'2015, ISBN: 978-1-4503-3473-0, DOI: \href{https://doi.org/10.1145/2740908.2742141}{https://doi.org/10.1145/2740908.2742141}. 	

%	\bibitem{DBLP:conf/semweb/NguyenTNS15} %\textbf{Paper in conference proceedings}:\\ 
%	\underline{Phuong T. Nguyen}, Paolo Tomeo, Tommaso Di Noia, Eugenio Di Sciascio: ``\emph{Content-based recommendations via DBpedia and Freebase: A case study in the music domain},'' in Proceedings of the 14th International Semantic Web Conference, ISWC 2015, ISBN: 978-3-319-25010-6, DOI: \href{https://doi.org/10.1007/978-3-319-25007-6\_35}{https://doi.org/10.1007/978-3-319-25007-6\_35}.  	
		
%	\bibitem{DBLP:conf/ictcc/NguyenLZ14} %\textbf{Paper in conference proceedings}:\\ 
%	\underline{Phuong T. Nguyen}, Hong Anh Le, Thomas Zinner ``\emph{A Context-Aware Traffic Engineering Model for Software-Defined Networks},'' in Proceedings of International Conference on Nature of Computation and Communication, ICTCC 2014, ISBN: 978-3-319-15392-6, DOI: \href{https://doi.org/10.1007/978-3-319-15392-6_8}{https://doi.org/10.1007/978-3-319-15392-6\_8}.

%	\bibitem{NguyenNICS2014} %\textbf{Paper in conference proceedings}:\\ 
%	\underline{Phuong T. Nguyen}, Loan T. Phan ``\emph{A Proposed Architecture for the Realization and Management of an Information-Centric Network,}'' in Proceedings of the first NAFOSTED Conference on Information and Computer Science, NICS 2014.
%		
%	\bibitem{DBLP:conf/iccasa/Nguyen13} %\textbf{Paper in conference proceedings}:\\ 
%	\underline{Phuong T. Nguyen}, ``\emph{Building Consensus in Context-Aware Systems Using Ben-Or’s Algorithm: Some Proposals for Improving the Convergence Speed},'' in Proceedings of the International Conference on Context-Aware Systems and Applications, ICCASA 2013, ISBN: 978-3-319-05939-6, DOI: \href{https://link.springer.com/chapter/10.1007/978-3-319-05939-6\_9}{https://link.springer.com/chapter/10.1007/978-3-319-05939-6\_9}.
%	
%	\bibitem{DBLP:conf/iccasa/NguyenSR13} %\textbf{Paper in conference proceedings}:\\ 
%	\underline{Phuong T. Nguyen}, Volkmar Schau, Wilhelm Rossak ``\emph{A Context-Aware Model for the Management of Agent Platforms in Dynamic Networks},'' in Proceedings of the International Conference on Context-Aware Systems and Applications, ICCASA 2013, ISBN: 978-3-319-05939-6, DOI: \href{https://link.springer.com/chapter/10.1007/978-3-319-05939-6_8}{https://link.springer.com/chapter/10.1007/978-3-319-05939-6\_8}.
%	
%	
%	\bibitem{EUMAS2012} %\textbf{Paper in conference proceedings}:\\ 
%	\underline{Phuong T. Nguyen}, Volkmar Schau, Wilhelm Rossak ``\emph{An Adaptive Communication Model for Mobile Agents inspired by the Honey Bee Colony: Theory and Evaluation},'' in Proceedings of the 10th European Workshop on Multiagent Systems, EUMAS 2012.
%	
%	\bibitem{EUMAS2011} %\textbf{Paper in conference proceedings}:\\ 
%	\underline{Phuong T. Nguyen}, Volkmar Schau, Wilhelm Rossak ``\emph{Towards an Adaptive Communication Model for Mobile Agents in Highly Dynamic Networks based on Swarming Behaviour},'' in Proceedings of the 9th European Workshop on Multi-agent Systems, EUMAS 2011.	
%	
%	\bibitem{NI2CS2011} %\textbf{Paper in conference proceedings}:\\ 
%	\underline{Phuong T. Nguyen}, Volkmar Schau, Wilhelm Rossak, ``\emph{Performance comparison of some message transport protocol implementations for agent community communication},'' in Proceedings of the 11th International Conference on Innovative Internet Community Systems, I2CS 2011, Lecture Notes in Informatics, ISBN: 978-3-88579-280-2.
%
%


%\subsection*{Workshop papers}



%	\bibitem{NguyenIIR2021}%\textbf{Paper in conference proceedings}:\\
%	\underline{Phuong T. Nguyen}, Juri Di Rocco, Riccardo Rubei, Claudio Di Sipio, Davide Di Ruscio, ``\emph{Recommending Third-party Library Updates with LSTM Neural Networks},'' in Proceedings of the 11th Italian Information Retrieval Workshop, September 13--15, 2021, Bari, Italy.
%
%	\bibitem{KaRS2021}%\textbf{Paper in conference proceedings}:\\
%	Juri Di Rocco, Claudio Di Sipio, Davide Di Ruscio, \underline{Phuong T. Nguyen}, Claudio Pomo, ``\emph{On the Need for a Body of Knowledge on Recommender Systems},'' in Proceedings of the 3rd Edition of Knowledge-aware and Conversational Recommender Systems (KaRS) \& 5th Edition of Recommendation in Complex Environments (ComplexRec) Joint Workshop @ RecSys 2021, September 27--October 1, 2021, Amsterdam, Netherlands.

%	\bibitem{OpenMBEE2021} %\textbf{Paper in conference proceedings}:\\
%	Riccardo Rubei, Juri Di Rocco, Davide Di Ruscio, \underline{Phuong T. Nguyen}, Alfonso Pierantonio, ``\emph{A Lightweight Approach for the Automated Classification and Clustering of Metamodels},'' in Proceedings of the 2nd International Workshop on Open Model Based Engineering Environment (OpenMBEE 2021), co-located with the IEEE/ACM 24th International Conference on Model Driven Engineering Languages and Systems, MODELS 2021, DOI: \href{https://doi.org/10.1109/MODELS-C53483.2021.00074}{https://doi.org/10.1109/MODELS-C53483.2021.00074}. 
%
%	\bibitem{10.1145/3417990.3420202} %\textbf{Paper in conference proceedings}:\\
%	Claudio Di Sipio, Davide Di Ruscio, \underline{Phuong T. Nguyen} ``\emph{Democratizing the development of recommender systems by means of low-code platforms},'' in Proceedings of the 23rd ACM/IEEE International Conference on Model Driven Engineering Languages and Systems: Companion Proceedings, MODELS 2020, DOI: \href{https://doi.org/10.1145/3417990.3420202}{https://doi.org/10.1145/3417990.3420202}.
%			
%	\bibitem{10.1145/3417990.3419626} %\textbf{Paper in conference proceedings}:\\
%	Angela Barriga,  Davide Di Ruscio, Ludovico Iovino, \underline{Phuong T. Nguyen}, Alfonso Pierantonio, ``\emph{An Extensible Tool-Chain for Analyzing Datasets of Metamodels},'' in Proceedings of the 23rd ACM/IEEE International Conference on Model Driven Engineering Languages and Systems: Companion Proceedings,  Association for Computing Machinery, DOI: \href{https://doi.org/10.1145/3417990.3419626}{https://doi.org/10.1145/3417990.3419626}.%  (\emph{The authors are listed in alphabetical order}). 	
%	
%	\bibitem{DBLP:conf/caise/NguyenRR19}%\textbf{Paper in conference proceedings}:\\
%	\underline{Phuong T. Nguyen}, Juri Di Rocco, Davide Di Ruscio ``\emph{Building Information Systems Using Collaborative-Filtering Recommendation Techniques},'' in Proceedings of Advanced Information Systems Engineering Workshops, CAiSE 2019, ISBN: 978-3-030-21290-2, DOI: \href{https://doi.org/10.1007/978-3-030-20948-3\_19}{https://doi.org/10.1007/978-3-030-20948-3\_19}.
	
%	\bibitem{NguyenKars2018} %\textbf{Paper in conference proceedings}:\\ 
%	\underline{Phuong T. Nguyen}, Juri Di Rocco, Davide Di Ruscio ``\emph{Knowledge-aware Recommender System for Software Development},'' in Proceedings of the Knowledge-aware and Conversational Recommender Systems Workshop, KaRS 2018 (co-located with RecSys 2018), October 7, 2018, Vancouver, Canada \href{http://ceur-ws.org/Vol-2290/kars2018_paper4.pdf}{(http://ceur-ws.org/Vol-2290/kars2018\_paper4.pdf)}.	
%	
%	\bibitem{NguyenIIR2018} %\textbf{Paper in conference proceedings}:\\
%	\underline{Phuong T. Nguyen}, Juri Di Rocco, Davide Di Ruscio ``\emph{Mining Software Repositories to Support OSS Developers: A Recommender Systems Approach},'' in Proceedings of the Italian Information Retrieval Workshop, IIR 2018 \href{http://ceur-ws.org/Vol-2140/paper9.pdf}{(http://ceur-ws.org/Vol-2140/paper9.pdf)}.
%	
%	\bibitem{10.1145/2740908.2742141} %\textbf{Paper in conference proceedings}:\\ 
%	\underline{Phuong T. Nguyen}, Paolo Tomeo, Tommaso Di Noia, Eugenio Di Sciascio: ``\emph{An evaluation of SimRank and Personalized PageRank to build a recommender system for the Web of Data},'' in Proceedings of the 7th International Workshop on Web Intelligence \& Communities, at WWW'2015, ISBN: 978-1-4503-3473-0, DOI: \href{https://doi.org/10.1145/2740908.2742141}{https://doi.org/10.1145/2740908.2742141}. 	
%	
%			
%	\bibitem{EUMAS2012} %\textbf{Paper in conference proceedings}:\\ 
%	\underline{Phuong T. Nguyen}, Volkmar Schau, Wilhelm Rossak ``\emph{An Adaptive Communication Model for Mobile Agents inspired by the Honey Bee Colony: Theory and Evaluation},'' in Proceedings of the 10th European Workshop on Multiagent Systems, EUMAS 2012.
%	
%	\bibitem{EUMAS2011} %\textbf{Paper in conference proceedings}:\\ 
%	\underline{Phuong T. Nguyen}, Volkmar Schau, Wilhelm Rossak ``\emph{Towards an Adaptive Communication Model for Mobile Agents in Highly Dynamic Networks based on Swarming Behaviour},'' in Proceedings of the 9th European Workshop on Multi-agent Systems, EUMAS 2011.	


%\subsection*{Journal First}
%
%	\bibitem{MODELS2022} Juri Di Rocco, Davide Di Ruscio, Claudio Di Sipio, \underline{Phuong T. Nguyen}, Alfonso Pierantonio, ``\emph{MemoRec: A Recommender System for Assisting Modelers in Specifying Metamodels},'' Springer Software and Systems Modeling (SoSyM), DOI: \href{https://doi.org/10.1007/s10270-022-00994-2}{https://doi.org/10.1007/s10270-022-00994-2}, presented at the 25th International Conference on Model Driven Engineering Languages and Systems, MODELS 2022.
%
%
%	\bibitem{SANER2021} %\textbf{Journal-first paper}:\\
%	Riccardo Rubei, Claudio Di Sipio, \underline{Phuong T. Nguyen}, Juri Di Rocco, Davide Di Ruscio, ``\emph{PostFinder: Mining Stack Overflow posts to support software developers},'' Elsevier Information and Software and Technology (IST), 2020, ISSN: 0950-5849, DOI: \href{https://doi.org/10.1016/j.infsof.2020.106367}{https://doi.org/10.1016/j.infsof.2020.106367}, presented at the 28th IEEE International Conference on Software Analysis, Evolution and Reengineering, SANER 2021.
%
%	\bibitem{ICSME2020} %\textbf{Journal-first paper}:\\
%	Andrea Capiluppi, Davide Di Ruscio, Juri Di Rocco, \underline{Phuong T. Nguyen}, Nemitari Ajienka, ``\emph{Detecting Java Software Similarities by using Different Clustering Techniques},'' Elsevier Information and Software Technology (IST), 2020, ISSN: 0950-5849, DOI: \href{https://doi.org/10.1016/j.infsof.2020.106279}{https://doi.org/10.1016/j.infsof.2020.106279}, presented at the 36th IEEE International Conference on Software Maintenance and Evolution, ICSME 2020.
%	
%
%\subsection*{Demos}
%
%	\bibitem{RecSys2021} Claudio Di Sipio, Juri Di Rocco, Davide Di Ruscio, \underline{Phuong T. Nguyen}, ``\emph{LEV4REC: A Low-Code Environment to Support the Development of Recommender Systems},'' demo at the 15th ACM Conference on Recommender Systems, RecSys 2021, DOI: \href{https://doi.org/10.1145/3460231.3478885}{https://doi.org/10.1145/3460231.3478885}
%	
%	
%
%\subsection*{Posters}
%
%%\begin{itemize}
%	\bibitem{} Claudio Di Sipio, Davide Di Ruscio, \underline{Phuong T. Nguyen}, ``\emph{Using the Low-code Paradigm to Support the Development of Recommender Systems},'' Poster session at the 23rd ACM/\-IEEE International Conference on Model Driven Engineering Languages and Systems, MODELS 2020.
%%	\bibitem{} \underline{Phuong T. Nguyen}, ``\emph{Performance Evaluation of Video SSIM Quality Metric on VQEG FR-TV Phase I Test Dataset},'' Poster session at the 13th International Student Conference on Electrical Engineering, POSTER 2009.
%%\end{itemize}
%
%%\subsection*{Manuscripts under submission}
%%    \bibitem{NguyenASE2021} \textbf{Conference manuscript}:\\ \underline{Phuong T. Nguyen}, Juri Di Rocco, Claudio Di Sipio, Davide Di Ruscio, Massimiliano Di Penta, ``\emph{Adversarial Attacks to API Recommender Systems: Time to Wake Up and Smell the Coffee?}'' to be submitted to the 36th IEEE/ACM International Conference on Automated Software Engineering, ASE 2021.
%
%%
%%\subsection*{Manuscripts under review/revision}   
%%	
%%	\bibitem{NguyenESWALUPE}  \textbf{Journal manuscript}:\\  %\textbf{Journal paper}:\\	
%%	\underline{Phuong T. Nguyen}, Claudio Di Sipio, Juri Di Rocco, Riccardo Rubei, Davide Di Ruscio, Massimiliano Di Penta ``\emph{Fitting Missing API Puzzles with Machine Translation Techniques},'' submitted to Elsevier Expert Systems with Applications (ESWA), under review. %IEEE Transactions on Software Engineering (TSE), 2021, ISSN: 1939-3520, DOI: \href{https://doi.org/10.1109/TSE.2021.3059907}{https://doi.org/10.1109/TSE.2021.3059907}, preprint: \href{https://arxiv.org/abs/2102.07508}{https://arxiv.org/abs/2102.07508}.
%%	
%%	\bibitem{DuongASOC2021} \textbf{Journal manuscript}:\\ Linh T. Duong, Cong Q. Chu, \underline{Phuong T. Nguyen}, Son T. Nguyen, Binh Q. Tran, ``\emph{Edge detection and graph neural networks to classify mammograms: A case study with a dataset from Vietnamese patients},'' submitted to Elsevier Applied Soft Computing (ASOC), under review.
%%	
%%	\bibitem{DuongASOC2021} \textbf{Journal manuscript}:\\ Linh T. Duong, \underline{Phuong T. Nguyen}, Ludovico Iovino, Michele Flammini, ``\emph{Automatic Detection of Covid-19 from Chest X-ray and Lung Computed Tomography Images using Deep Neural Networks and Transfer Learning},'' submitted to Elsevier Applied Soft Computing (ASOC), under review after revision, preprint in medRxiv, DOI: \href{https://doi.org/10.1101/2020.08.13.20173997}{10.1101/2020.08.13.20173997}.
%%
%%
%%%	\bibitem{NguyenJSS2021} \textbf{Journal manuscript}:\\
%%%	\underline{Phuong T. Nguyen}, Juri Di Rocco, Riccardo Rubei, Claudio Di Sipio, Davide Di Ruscio, ``\emph{DeepLib: Recommending Third-party Library Updates with Deep Neural Networks},'' submitted to Elsevier Expert Systems with Applications (ESWA), under review.
%%	
%%	\bibitem{DiSipio2022} \textbf{Journal manuscript}:\\
%%	Claudio Di Sipio, Juri Di Rocco, Davide Di Ruscio, \underline{Phuong T. Nguyen}, ``\emph{MORGAN: An intelligent modeling assistant based on kernel
%%		similarity and graph neural networks},'' an invited paper submitted to the Special Issue for MODELS 2021, Springer Software and Systems Modeling (SoSyM), under review.
%%	
%%	\bibitem{DiRuscio2022} \textbf{Journal manuscript}:\\
%%	Davide Di Ruscio, Paola Inverardi, Patrizio Migliarini, \underline{Phuong T. Nguyen}, ``\emph{Leveraging Privacy Profiles to Empower Users in the Digital Society},'' submitted to Elsevier Information Sciences (INS), under review.
%%
%
%%	
%
%%	
%%%	\bibitem{DuongESWA2021} \textbf{Journal manuscript}:\\ Linh T. Duong, Nhi H. Le, Toan B. Tran, Vuong M. Ngo, \underline{Phuong T. Nguyen}, ``\emph{Detection of Tuberculosis from Chest X-ray Images: Boosting the Performance with Vision Transformer and Transfer Learning},'' submitted to Elsevier Expert Systems with Applications (ESWA), under revision.
%%%\subsection*{Manuscripts under review}
%%
%%    %the 25th International Conference on Evaluation and Assessment in Software Engineering, EASE 2021, under review.
%%
%%	\bibitem{DiRoccoEMSE2021} \textbf{Journal manuscript}:\\ Juri Di Rocco, Davide Di Ruscio, Claudio Di Sipio, \underline{Phuong T. Nguyen}, Riccardo Rubei, ``\emph{HybridRec: A Recommender System for Tagging GitHub Repositories},'' submitted to Springer Applied Intelligence (APIN), under review.
%%	%to Springer Empirical Software Engineering (EMSE), under review.
%%	
%%%	\bibitem{DiRoccoSoSyM2021} \textbf{Journal manuscript}:\\ Juri Di Rocco, Davide Di Ruscio, Claudio Di Sipio, \underline{Phuong T. Nguyen}, Alfonso Pierantonio, ``\emph{MemoRec: A Recommender System for Assisting Modelers in Specifying Metamodels},'' submitted to Springer Software and Systems Modeling (SoSyM), under review after the revision.
%%	
%%	
%%%\end{itemize}
%%
%%%\subsection*{Preprints}
%%%	\bibitem{DuongmedRxiv} Linh T. Duong, \underline{Phuong T. Nguyen}, Ludovico Iovino, and Michele Flammini, ``\emph{Deep Learning for Automatic Detection of Covid-19 from Chest X-ray Images},'' medRxiv, DOI: \href{https://doi.org/10.1101/2020.08.13.20173997}{10.1101/2020.08.13.20173997}.
%%
%%
%%\vspace{0.5cm} 

%\emph{Phuong T. Nguyen authorizes the processing of personal data pursuant to Law 676/96 and its subsequent amendments and / or additions.}\\
\emph{L'Aquila, December 18$^{th}$ 2023}
%\emph{L'Aquila, November 12$^{th}$ 2023}
%\emph{L'Aquila, November 3$^{rd}$ 2023}
%\emph{L'Aquila, October 7$^{th}$ 2023}
%\emph{L'Aquila, September 27$^{th}$ 2023}
%\emph{L'Aquila, September 15$^{th}$ 2023}
%\emph{L'Aquila, August 4$^{th}$ 2023}
%\emph{L'Aquila, May 17$^{th}$ 2023}
%\emph{L'Aquila, April 21$^{st}$ 2023}
%\emph{L'Aquila, April 9$^{th}$ 2023}
%\emph{L'Aquila, December 30$^{th}$ 2022}
%\emph{L'Aquila, November 24$^{th}$ 2022}
%\emph{L'Aquila, November 16$^{th}$ 2022}
%\emph{L'Aquila, July 12$^{th}$ 2022}
%\emph{L'Aquila, May 12$^{th}$ 2022}
%\emph{L'Aquila, May 5$^{th}$ 2022}
%\emph{L'Aquila, April 15$^{th}$ 2022}
%\emph{L'Aquila, March 26$^{th}$ 2022}
%\emph{L'Aquila, March 23$^{rd}$ 2022}
%\vspace{1cm} 

%\begin{figure}[h!]
%	\vspace{-.2cm}
%	\begin{tabular}{c c}
%	 & \subfigure{\includegraphics[width=0.30\columnwidth]{Signature.png}} \\
%	\end{tabular}	
%	\vspace{-.6cm}
%\end{figure}

\textbf{Phuong T. Nguyen}
		
\end{thebibliography}



\end{document}